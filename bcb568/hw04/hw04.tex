\documentclass[a4paper, 10pt]{article}
\usepackage[latin1]{inputenc}          % Accept european-encoded (latin1) characters.
\usepackage{a4wide}                    % Wide paper
\usepackage[parfill]{parskip}          % skip a line instead of indenting new paragraphs
\usepackage{graphicx}                  % For eps figures
\usepackage{epsfig}                    % Alternative package
\usepackage{hyperref}
\hypersetup
{
  colorlinks=false,
  pdfborder={0,0,0},
}
 
\usepackage{fancyhdr}
\fancyhead[L]{\class \;- \assignment }
\fancyhead[R]{\author }
\renewcommand{\footrulewidth}{0.5pt} % Insert a line above the footer
\pagestyle{fancy}
\usepackage[hang,small,bf]{caption}
\usepackage{palatino}
\usepackage{amsmath}
\usepackage{amssymb}
\usepackage{enumerate}
 
% convenience commands
\renewcommand{\author}{Daniel Standage}
\newcommand{\class}{BCB 568}
\newcommand{\instructor}{Brendel/Dorman}
\newcommand{\assignment}{HW 4}
\newcommand{\duedate}{Feb ?, 2011}
 
\newcounter{prob_num}
\setcounter{prob_num}{1}
% usage: \problem
\newcommand{\problem}{\vspace{20pt}\arabic{prob_num}.\stepcounter{prob_num}\par}
% usage: \head{name}{class}{assignment}
\newcommand{\head}{\begin{center}\begin{tabular*}{\linewidth}{l@{\extracolsep{\fill}}r} & \class \;- \assignment \\ & \duedate \end{tabular*}\end{center} \hfill }
% usage: \eqn{equation}{label}
\newcommand{\eqn}[2]{\begin{equation}#1\label{#2}\end{equation}}
 
\begin{document}

%%%%%%%%%%%%%%%%%%%%%%%%%%%%%%%%%%%%%%%%%%%%%%%%%%
\problem

The probability of the output sequence is shown below, as well as the derivations corresponding to the forward and reverse algorithms.
\[ P_{\lambda}(RYYR) = \frac{101}{1250} \]

\subsection*{Forward algorithm}

The following table shows all values of $E_i$ and $I_i$ as calculated by the forward algorithm (calculations shown below). Once the table is completely filled out, the probability of the sequence $RYYR$ can be determined as the sum of $E_N$ and $I_N$.
\[ P_{\lambda}(RYYR) = E_N + I_N = E_4 + I_4 = \frac{31}{625} + \frac{39}{1250} = \frac{101}{1250} \]

\renewcommand{\arraystretch}{2}
\begin{center}
\begin{tabular}{|c|c c c c|}
  \hline
  \textbf{E} & $\frac{3}{5}$  & $\frac{1}{25}$ & $\frac{4}{125}$  & $\frac{31}{625}$  \\
  \textbf{I} & $\frac{1}{10}$ & $\frac{3}{10}$ & $\frac{27}{250}$ & $\frac{39}{1250}$ \\
  \hline
             & 1 & 2 & 3 & 4 \\
  \hline
\end{tabular}
\end{center}
\renewcommand{\arraystretch}{1}

\[ E_1 = \pi_{E}P(R|E) = \frac{3}{4}\cdot\frac{4}{5} = \frac{3}{5} \]
\[ I_1 = \pi_{I}P(R|I) = \frac{1}{4}\cdot\frac{2}{5} = \frac{1}{10} \]
\[ E_2 = \left[ E_{1}\tau_{EE}+ I_{1}\tau_{IE} \right]P(Y|E) = \left[ \frac{3}{5}\cdot\frac{1}{4} + \frac{1}{10}\cdot\frac{1}{2} \right]\frac{1}{5} = \frac{1}{25} \]
\[ I_2 = \left[ I_{1}\tau_{II}+ E_{1}\tau_{EI} \right]P(Y|I) = \left[ \frac{1}{10}\cdot\frac{1}{2} + \frac{3}{5}\cdot\frac{3}{4} \right]\frac{3}{5} = \frac{3}{10} \]
\[ E_3 = \left[ E_{2}\tau_{EE}+ I_{2}\tau_{IE} \right]P(Y|E) = \left[ \frac{1}{25}\cdot\frac{1}{4} + \frac{3}{10}\cdot\frac{1}{2} \right]\frac{1}{5} = \frac{4}{125} \]
\[ I_3 = \left[ I_{2}\tau_{II}+ E_{2}\tau_{EI} \right]P(Y|I) = \left[ \frac{3}{10}\cdot\frac{1}{2} + \frac{1}{25}\cdot\frac{3}{4} \right]\frac{3}{5} = \frac{27}{250} \]
\[ E_4 = \left[ E_{3}\tau_{EE}+ I_{3}\tau_{IE} \right]P(R|E) = \left[ \frac{4}{125}\cdot\frac{1}{4} + \frac{27}{250}\cdot\frac{1}{2} \right]\frac{4}{5} = \frac{31}{625} \]
\[ I_4 = \left[ I_{3}\tau_{II}+ E_{3}\tau_{EI} \right]P(R|I) = \left[ \frac{27}{250}\cdot\frac{1}{2} + \frac{4}{125}\cdot\frac{3}{4} \right]\frac{2}{5} = \frac{39}{1250} \]


\subsection*{Backward algorithm}

The following table shows all the values of $\bar{E}_i$ and $\bar{I}_i$ as calculated by the backward algorithm. Once the table is completely filled out, we can use the following identity to find the output probability.
\[ P(O_1 O_2 ... O_N) = \sum_{\{Q\}}P(O_1 O_2 ... O_N | Q_1)P(Q_1) \]
If we recognize that all $O_i$ are independent, we can pull out $O_1$ to obtain
\[ P(O_1 O_2 ... O_N) = \sum_{\{Q\}}P(O_2 O_3 ... O_N | Q_1)P(O_1|Q_1)P(Q_1) \]
Returning to the definition of $\bar{E}_{i}$ and $\bar{I}_{i}$, we can compute this sum as follows.
\[ \sum_{\{Q\}}P(O_2 O_3 ... O_N | Q_1)P(O_1|Q_1)P(Q_1) = \bar{E}_{1}P(R|E)P(E) + \bar{I}_{1}P(R|I)P(I) = \frac{473}{4000}\cdot\frac{4}{5}\cdot\frac{3}{4} + \frac{197}{2000}\cdot\frac{2}{5}\cdot\frac{1}{4} = \frac{101}{1250} \]


\renewcommand{\arraystretch}{2}
\begin{center}
\begin{tabular}{|c|c c c c|}
  \hline
  \textbf{E} & $\frac{473}{4000}$ & $\frac{59}{200}$ & $\frac{1}{2}$ & $1$  \\
  \textbf{I} & $\frac{197}{2000}$ & $\frac{23}{100}$ & $\frac{3}{5}$ & $1$ \\
  \hline
             & 1 & 2 & 3 & 4 \\
  \hline
\end{tabular}
\end{center}
\renewcommand{\arraystretch}{1}

\[ \bar{E}_4 = \bar{I}_4 = 1 \]
\[ \bar{E}_3 = \tau_{EE}\bar{E}_{4}P(R|E) + \tau_{EI}\bar{I}_{4}P(R|I) = \frac{1}{4}\cdot 1 \cdot\frac{4}{5} + \frac{3}{4}\cdot 1 \cdot\frac{2}{5} = \frac{1}{2} \]
\[ \bar{I}_3 = \tau_{II}\bar{I}_{4}P(R|I) + \tau_{IE}\bar{E}_{4}P(R|E) = \frac{1}{2}\cdot 1 \cdot\frac{2}{5} + \frac{1}{2}\cdot 1 \cdot\frac{4}{5} = \frac{3}{5} \]
\[ \bar{E}_2 = \tau_{EE}\bar{E}_{3}P(Y|E) + \tau_{EI}\bar{I}_{3}P(Y|I) = \frac{1}{4}\cdot\frac{1}{2}\cdot\frac{1}{5} + \frac{3}{4}\cdot\frac{3}{5}\cdot\frac{3}{5} = \frac{59}{200} \]
\[ \bar{I}_2 = \tau_{II}\bar{I}_{3}P(Y|I) + \tau_{IE}\bar{E}_{3}P(Y|E) = \frac{1}{2}\cdot\frac{3}{5}\cdot\frac{3}{5} + \frac{1}{2}\cdot\frac{1}{2}\cdot\frac{1}{5} = \frac{23}{100} \]
\[ \bar{E}_1 = \tau_{EE}\bar{E}_{2}P(Y|E) + \tau_{EI}\bar{I}_{2}P(Y|I) = \frac{1}{4}\cdot\frac{59}{200}\cdot\frac{1}{5} + \frac{3}{4}\cdot\frac{23}{100}\cdot\frac{3}{5} = \frac{473}{4000} \]
\[ \bar{I}_1 = \tau_{II}\bar{I}_{2}P(Y|I) + \tau_{IE}\bar{E}_{2}P(Y|E) = \frac{1}{2}\cdot\frac{23}{100}\cdot\frac{3}{5} + \frac{1}{2}\cdot\frac{59}{200}\cdot\frac{1}{5} = \frac{197}{2000} \]


%%%%%%%%%%%%%%%%%%%%%%%%%%%%%%%%%%%%%%%%%%%%%%%%%%
\problem

The probability is shown below, along with its derivation.
\[ P_{\lambda}(Q_2 = E|RYY) = \frac{1}{7} \]

To derive this probability generally, we use the relationship between joint and conditional probabilities $P(A, B) = P(B|A)P(A)$, the independence of $O_i$ and $O_j$ for all $i \neq j$, and the recurrences from the forward and backward algorithms.
\renewcommand{\arraystretch}{1.75}
\begin{eqnarray}
P(Q_{i}=E|O_1 O_2...O_N) &=& \frac{P(O_1 O_2...O_N, Q_{i} = E)}{P(O_1 O_2...O_N)}                                                     \nonumber \\
                         &=& \frac{P(O_1 O_2...O_i, Q_{i} = E, O_{i+1} O_{i+2} ... O_N)}{P(O_1 O_2...O_N)}                            \nonumber \\
                         &=& \frac{P(O_{i+1} O_{i+2} ... O_N|O_1 O_2...O_i, Q_{i} = E)P(O_1 O_2...O_i, Q_{i} = E)}{P(O_1 O_2...O_N)}  \nonumber \\
                         &=& \frac{P(O_{i+1} O_{i+2} ... O_N|Q_{i} = E)P(O_1 O_2...O_i, Q_{i} = E)}{P(O_1 O_2...O_N)}                 \nonumber \\
                         &=& \frac{\bar{E}_i \cdot E_i}{P(O_1 O_2...O_N)}                                                             \nonumber \\
                         &=& \frac{\bar{E}_i \cdot E_i}{E_N + I_N}                                                                    \nonumber
\end{eqnarray}
\renewcommand{\arraystretch}{1}

Taking probabilities calculated by the forward and backward algorithms (shown below), we can find the desired probability as follows.
\[ P_{\lambda}(Q_2 = E|RYY) = \frac{\bar{E}_2 \cdot E_2}{E_3 + I_3} = \frac{\frac{1}{2} \cdot \frac{1}{25}}{\frac{4}{125} + \frac{27}{250}} = \frac{1}{7} \]

\renewcommand{\arraystretch}{2}
\begin{center}
\begin{tabular}{|c|c c c|}
  \hline
  \textbf{E} & $\frac{3}{5}$  & $\frac{1}{25}$ & $\frac{4}{125}$  \\
  \textbf{I} & $\frac{1}{10}$ & $\frac{3}{10}$ & $\frac{27}{250}$ \\
  \hline
             & 1 & 2 & 3 \\
  \hline
\end{tabular}
\begin{tabular}{|c|c c c|}
  \hline
  \textbf{E} & $\frac{59}{200}$ & $\frac{1}{2}$ & $1$  \\
  \textbf{I} & $\frac{23}{100}$ & $\frac{3}{5}$ & $1$ \\
  \hline
             & 1 & 2 & 3 \\
  \hline
\end{tabular}
\end{center}
\renewcommand{\arraystretch}{1}


%%%%%%%%%%%%%%%%%%%%%%%%%%%%%%%%%%%%%%%%%%%%%%%%%%
\problem

To verify this value using the rules of total probability, we must build a table $T$ where $T_{i,j}$ is the joint probability of the $i^{\text{th}}$ output sequence and the $j^{\text{th}}$ state sequence. Each value in this table is calculated as follows. \[ T_{i,j} = P(O_1|Q_1)P(O_2|Q_2)P(O_3|Q_3)\pi_{Q_1}\tau_{Q_1 Q_2}\tau_{Q_2 Q_3} \] One row of this table $T$ (the only row relevant to this question) is shown below, transposed.

\renewcommand{\arraystretch}{2}
\begin{center}
\begin{tabular}{|c|c|}
  \hline
               & \textbf{RYY} \\
  \hline
  \textbf{EEE} & $\frac{4}{5}\cdot\frac{1}{5}\cdot\frac{1}{5}\cdot\frac{3}{4}\cdot\frac{1}{4}\cdot\frac{1}{4} = \frac{3}{2000} $ \\
  \textbf{EEI} & $\frac{4}{5}\cdot\frac{1}{5}\cdot\frac{3}{5}\cdot\frac{3}{4}\cdot\frac{1}{4}\cdot\frac{3}{4} = \frac{27}{2000}$ \\
  \textbf{EIE} & $\frac{4}{5}\cdot\frac{3}{5}\cdot\frac{1}{5}\cdot\frac{3}{4}\cdot\frac{3}{4}\cdot\frac{1}{2} = \frac{27}{1000}$ \\
  \textbf{EII} & $\frac{4}{5}\cdot\frac{3}{5}\cdot\frac{3}{5}\cdot\frac{3}{4}\cdot\frac{3}{4}\cdot\frac{1}{2} = \frac{81}{1000}$ \\
  \textbf{IEE} & $\frac{2}{5}\cdot\frac{1}{5}\cdot\frac{1}{5}\cdot\frac{1}{4}\cdot\frac{1}{2}\cdot\frac{1}{4} = \frac{1}{2000} $ \\
  \textbf{IEI} & $\frac{2}{5}\cdot\frac{1}{5}\cdot\frac{3}{5}\cdot\frac{1}{4}\cdot\frac{1}{2}\cdot\frac{3}{4} = \frac{9}{2000} $ \\
  \textbf{IIE} & $\frac{2}{5}\cdot\frac{3}{5}\cdot\frac{1}{5}\cdot\frac{1}{4}\cdot\frac{1}{2}\cdot\frac{1}{2} = \frac{3}{1000} $ \\
  \textbf{III} & $\frac{2}{5}\cdot\frac{3}{5}\cdot\frac{3}{5}\cdot\frac{1}{4}\cdot\frac{1}{2}\cdot\frac{1}{2} = \frac{9}{1000} $ \\
  \hline
\end{tabular}
\end{center}
\renewcommand{\arraystretch}{1}

The desired probability is obtained by summing up the all probabilities for which $Q_2 = E$ and then dividing by the probability of the output sequence (which is the sum of all probabilities in the row). Thus we have
\[ P_{\lambda}(Q_2 = E|RYY) = \frac{\frac{3}{2000}+\frac{27}{2000}+\frac{1}{2000}+\frac{9}{2000}}{\frac{3}{2000}+\frac{27}{2000}+\frac{27}{1000}+\frac{81}{1000}+\frac{1}{2000}+\frac{9}{2000}+\frac{3}{1000}+\frac{9}{1000}} = \frac{1}{7} \]



%%%%%%%%%%%%%%%%%%%%%%%%%%%%%%%%%%%%%%%%%%%%%%%%%%
\problem

The desired probability and corresponding derivation is shown below.
\[ P_{\lambda}(Q_2=E, Q_3=I|RYYR) = \frac{27}{202} \]

We can obtain the probability of the prefix using $E_2$ and the probability of the suffix using $\bar{I}_3$. We multiply these probabilities with the appropriate transition and output probabilities and then normalize by the probability of the output sequence to obtain the following formula (and solution using values calculated in problem 1).
\[ P_{\lambda}(Q_2=E, Q_3=I|RYYR) = \frac{E_2 \cdot \tau_{EI} \cdot P(Y|I) \cdot \bar{I}_3}{E_4 + I_4} = \frac{\frac{1}{25}\cdot\frac{3}{4}\cdot\frac{3}{5}\cdot\frac{3}{5}}{\frac{31}{625}+\frac{39}{1250}} = \frac{27}{202} \]


%%%%%%%%%%%%%%%%%%%%%%%%%%%%%%%%%%%%%%%%%%%%%%%%%%
\problem

The most probable state sequence given the output sequence $RYYR$ is $EIIE$. To derive this solution, we first fill out a table using the Viterbi algorithm. We then start at the maximum value in the last column of the table and begin our traceback to obtain the maximum probability state sequence.

\renewcommand{\arraystretch}{2}
\begin{center}
\begin{tabular}{|c|c|c|c|c|}
  \hline
  \textbf{E} & $\frac{4}{5}\cdot\frac{3}{4}=\frac{3}{5}$  & $\frac{3}{4}\cdot\frac{1}{4}\cdot\frac{1}{5}=\frac{3}{100}$ (from $e_1$)  & $\frac{27}{100}\cdot\frac{1}{2}\cdot\frac{1}{5}=\frac{27}{1000}$ (from $i_2$) & $\frac{81}{100}\cdot\frac{1}{2}\cdot\frac{4}{5}=\frac{81}{2500}$ (from $i_3$)  \\
  \textbf{I} & $\frac{2}{5}\cdot\frac{1}{4}=\frac{1}{10}$ & $\frac{3}{5}\cdot\frac{3}{4}\cdot\frac{4}{5}=\frac{27}{100}$ (from $e_1$) & $\frac{27}{100}\cdot\frac{1}{2}\cdot\frac{3}{5}=\frac{81}{1000}$ (from $i_2$) & $\frac{81}{100}\cdot\frac{1}{2}\cdot\frac{2}{5}=\frac{81}{5000}$ (from $i_3$) \\
  \hline
             & 1 & 2 & 3 & 4 \\
  \hline
\end{tabular}
\end{center}
\renewcommand{\arraystretch}{1}


\end{document}