\documentclass[a4paper, 10pt]{article}
\usepackage[latin1]{inputenc}          % Accept european-encoded (latin1) characters.
\usepackage{a4wide}                    % Wide paper
\usepackage[parfill]{parskip}          % skip a line instead of indenting new paragraphs
\usepackage{graphicx}                  % For eps figures
\usepackage{epsfig}                    % Alternative package
\usepackage{hyperref}
\hypersetup
{
  colorlinks=false,
  pdfborder={0,0,0},
}
 
\usepackage{fancyhdr}
\fancyhead[L]{\class \;- \assignment }
\fancyhead[R]{\author }
\renewcommand{\footrulewidth}{0.5pt} % Insert a line above the footer
\pagestyle{fancy}
\usepackage[hang,small,bf]{caption}
\usepackage{palatino}
\usepackage{amsmath}
\usepackage{amssymb}
\usepackage{enumerate}
 
% convenience commands
\renewcommand{\author}{Daniel Standage}
\newcommand{\class}{BCB 568}
\newcommand{\instructor}{Brendel/Dorman}
\newcommand{\assignment}{HW 6}
\newcommand{\duedate}{Feb ?, 2011}
 
\newcounter{prob_num}
\setcounter{prob_num}{1}
% usage: \problem
\newcommand{\problem}{\vspace{20pt}\arabic{prob_num}.\stepcounter{prob_num}\par}
% usage: \head{name}{class}{assignment}
\newcommand{\head}{\begin{center}\begin{tabular*}{\linewidth}{l@{\extracolsep{\fill}}r} & \class \;- \assignment \\ & \duedate \end{tabular*}\end{center} \hfill }
% usage: \eqn{equation}{label}
\newcommand{\eqn}[2]{\begin{equation}#1\label{#2}\end{equation}}
 
\begin{document}

%%%%%%%%%%%%%%%%%%%%%%%%%%%%%%%%%%%%%%%%%%%%%%%%%%
\problem

For $n$ OTUs, the number of pairwise distances $D_n$ between OTUs is ${n \choose 2}$.

For 2 OTUs, there is a single branch connecting them. For each additional OTU, a new bifurcation is created, introducing 2 additional branches. Therefore the number of branches $L_n$ for a tree with $n$ OTUs is $2n-3$.

For $n = 2,3$, there is a unique topological arrangement of the $n$ OTUs. For $n > 3$, however, an additional OTU can branch off from any existing branch in the topology. We use this fact to prove the number of distinct topologies for $n \geq 3$ OTUs by induction.

Let $T_n$ be the number of distinct topologies for $n$ OTUs. First, we consider $n = 3$. There is a unique topological arrangement of 3 OTUs, so $T_3 = 1$. Now let us assume that we have a tree with $k$ OTUs (and by our induction hypothesis, $T_k = \frac{(2k - 5)!}{2^{k-3}(k-3)!}$ distinct topologies and $L_k = 2k - 3$ branches). If we want to create a tree with $k+1$ OTUs, we simply add an OTU to one of the $L_k$ branches on one of the $T_k$ trees. Therefore, the number of distinct topologies for a tree with $k+1$ OTUs is
\[ T_k \cdot L_k = \frac{(2k - 5)!}{2^{k-3}(k-3)!} \cdot 2k - 3 = ... = \frac{(2(k+1) - 5)!}{2^{(k+1)-3}((k+1)-3)!} \]

%Each of these topologies has $L_n = 2n-3$ branches, and adding an additional OTU to any of these branches will give us a distinct topology with $k+1$ OTUs. Therefore, the number of distinct topologies with $k+1$ OTUs $T_{k+1} = T_{k} \cdot L_{k}$. ???

The table below shows the growth of $D_n$, $L_n$, and $T_n$ as $n$ grows.
\renewcommand{\arraystretch}{1.5}
\begin{center}
\begin{tabular}{|c|c|c|c|}
  \hline
  \textbf{Description} & \textbf{Symbol} & \textbf{Pattern of growth} & \textbf{Solution}\\
  \hline
  \# species    & $n$   & $2,3,4,5,...,n$                              & $n+1$           \\
  \# distances  & $D_n$ & $1,3,6,10,..., {n \choose 2}$                & $D_n + n$       \\
  \# branches   & $L_n$ & $1,3,5,7,..., 2n-3$                          & $L_n + 2$       \\
  \# topologies & $T_n$ & $1,1,3,...,1\cdot3\cdot5\cdot...\cdot(2n-5)$ & $T_n \cdot L_n$ \\
  \hline
\end{tabular}
\end{center}
\renewcommand{\arraystretch}{1}


%%%%%%%%%%%%%%%%%%%%%%%%%%%%%%%%%%%%%%%%%%%%%%%%%%
\problem

Let $q$ represent the fraction of identical sites in a pairwise sequence alignment, and let $d$ represent the number of substitutions per site. These values are related by the following simple equation. \[ q = 1 - d \]

However, we want to correct for unobserved substitutions. We would like to find a function $f(q) : [0,1] \rightarrow \mathbb{R}$ such that $f(q)$ is close to 0 as $q$ approaches 1, but is greater than $1 - q$ as $q$ approaches 0.

Let us proceed by defining $\lambda$ as the substitution (or mutation) rate for the sequences, and $\Delta t$ as some small time interval. The probability that there are no substitutions at a given site in time $\Delta t$ is \[ P_0(\Delta t) = 1 - \lambda \Delta t \] If we assume independence of disjoint time intervals, we can write the following probability for any arbitrary time interval. \[ P_0(t + \Delta t) = P_0(t)P_0(\Delta t) = P_0(t)[1 - \lambda \Delta t] = P_0(t) - P_0(t)\lambda \Delta t \]
\[ P_0(t + \Delta t) - P_0(t) = - P_0(t)\lambda \Delta t \]
\[ \frac{P_0(t + \Delta t) - P_0(t)}{\Delta t} = - \lambda P_0(t) \]

As $\Delta t$ becomes arbitrarily small, this gives us the differential equation \[ P'(t) = -\lambda P_0(t) \] with solution \[ P_0(t) = e^{-\lambda t} \]

Therefore, if $d = \lambda \Delta t$ and $q = e^{\lambda \Delta t}$, then the approximating function we desire can be written as follows. \[ d \approx f(q) = -ln(q) \]


%%%%%%%%%%%%%%%%%%%%%%%%%%%%%%%%%%%%%%%%%%%%%%%%%%
\problem
First we show that calculating $G_{i,j}$ is a constant time operation. There are two possible ways to have a gap ending with an arbitrary $G_{i,j}$: it is either the extension of a previously opened gap, or the opening of a new gap. For these two cases, the gap can be introduced/extended in either of the two sequences. Therefore we can rewrite the recurrence for $G_{i,j}$ as follows.
\[
G_{ij} = \max \left\{
  \begin{array}{l}
    G_{i-1,j} - \beta \\
    G_{i,j-1} - \beta \\
    S_{i-1,j} + w(1)  \\
    S_{i,j-1} + w(1)  \\
  \end{array}
  \right.
\]
So calculating $G_{i,j}$ requires 4 operations. Calculating $D_{i,j}$ is also constant time, requiring only a single operation.

Recall the given recurrence for $S_{i,j} = \max \{ D_{i,j}, G_{i,j} \}$. Therefore, for two sequences of lengths $m$ and $n$, calculating $S_{mn}$ will require $O(5mn) \in O(mn)$ operations.

\end{document}