\documentclass[a4paper, 10pt]{article}
\usepackage[latin1]{inputenc}          % Accept european-encoded (latin1) characters.
\usepackage{a4wide}                    % Wide paper
\usepackage[parfill]{parskip}          % skip a line instead of indenting new paragraphs
\usepackage{graphicx}                  % For eps figures
\usepackage{epsfig}                    % Alternative package
\usepackage{hyperref}
\hypersetup
{
  colorlinks=false,
  pdfborder={0,0,0},
}
 
\usepackage{fancyhdr}
\fancyhead[L]{\class \;- \assignment }
\fancyhead[R]{\author }
\renewcommand{\footrulewidth}{0.5pt} % Insert a line above the footer
\pagestyle{fancy}
\usepackage[hang,small,bf]{caption}
\usepackage{palatino}
\usepackage{amsmath}
\usepackage{amssymb}
\usepackage{enumerate}
 
% convenience commands
\renewcommand{\author}{Daniel Standage}
\newcommand{\class}{BCB 568}
\newcommand{\instructor}{Brendel/Dorman}
\newcommand{\assignment}{HW 2}
\newcommand{\duedate}{Feb 3, 2011}
 
\newcounter{prob_num}
\setcounter{prob_num}{1}
% usage: \problem
\newcommand{\problem}{\vspace{20pt}\arabic{prob_num}.\stepcounter{prob_num}\par}
% usage: \head{name}{class}{assignment}
\newcommand{\head}{\begin{center}\begin{tabular*}{\linewidth}{l@{\extracolsep{\fill}}r} & \class \;- \assignment \\ & \duedate \end{tabular*}\end{center} \hfill }
% usage: \eqn{equation}{label}
\newcommand{\eqn}[2]{\begin{equation}#1\label{#2}\end{equation}}
 
\begin{document}

\section*{Examples}
%%%%%%%%%%%%%%%%%%%%%%%%%%%%%%%%%%%%%%%%%%%%%%%%%%
\problem

\subsection*{Poisson distribution}
Using the $pmf$ of the Poisson distribution, we obtain the following probability generating function.
\begin{eqnarray}
  A(s) &=& \sum_{k=0}^{\infty}\frac{\lambda^k}{k!}e^{-\lambda}s^k \nonumber \\
       &=& e^{-\lambda}\left( 1 + \frac{\lambda^1 s^1}{1!} + \frac{\lambda^2 s^2}{2!} + ... \right) \nonumber \\
       &=& e^{-\lambda}\left( 1 + \frac{(\lambda s)^1}{1!} + \frac{(\lambda s)^2}{2!} + ... \right) \nonumber \\
       &=& e^{-\lambda}\left( e^{\lambda s} \right) = e^{\lambda s - \lambda} \nonumber \\
       &=& e^{\lambda(s - 1)}
\end{eqnarray}

Taking the first and second derivatives of this $pgf$ we obtain the following functions.
\begin{eqnarray}
  A'(s) &=& \lambda s e^{\lambda(s - 1)}
\end{eqnarray}
\begin{eqnarray}
  A''(s) &=& (\lambda s)' e^{\lambda(s - )} + \lambda s (e^{\lambda(s - 1)})' \nonumber \\
         &=& \lambda e^{\lambda(s - 1)} + \lambda^2 s^2 e^{\lambda(s -1)}     \nonumber \\
         &=& e^{\lambda(s -1)} \left( \lambda^2 s^2 + \lambda \right)
\end{eqnarray}

Using functions (2) and (3) we can calculate the mean and variance for the Poisson distribution.
\begin{eqnarray}
  E &=& A'(1) = \lambda \cdot 1 \cdot 1 = \lambda
\end{eqnarray}
\begin{eqnarray}
  Var &=& A''(1) + A'(1) - A'(1)^2 \nonumber \\
      &=& (\lambda^2 + \lambda) + (\lambda) - (\lambda)^2 \nonumber \\
      &=& 2\lambda
\end{eqnarray}

\subsection*{Binomial distribution}
Using the $pmf$ of the binomial distribution, we obtain the following probability generating function.
\begin{eqnarray}
  A(s) &=& \sum_{k=0}^{\infty}\binom{n}{k}p^{k}q^{n-k}s^k \nonumber \\
       &=& \sum_{k=0}^{\infty}\binom{n}{k}(ps)^{k}q^{n-k} \nonumber \\
       &=& q^n + \binom{n}{1}(ps)^1 q^{n-1} + \binom{n}{2}(ps)^2 q^{n-2} + ... \nonumber \\
       &=& (ps + q)^n
\end{eqnarray}

Taking the first and second derivatives of this $pgf$ we obtain the following functions.
\begin{eqnarray}
A'(s) &=& np(ps + q)^{n-1}
\end{eqnarray}
\begin{eqnarray}
A''(s) &=& n p^2(n-1)(ps + q)^{n-2}
\end{eqnarray}

Using functions 7 and 8 we can calculate the mean and variance for the binomial distribution.
\begin{eqnarray}
E &=& A'(1) = pn (p + q)^{n-1} \nonumber \\
  &=& np (1)^{n-1} \nonumber \\
  &=& np
\end{eqnarray}
\begin{eqnarray}
Var &=& A''(1) + A'(1) - A'(1)^2 \nonumber \\
    &=& n p^2 (n-1) + np - n^2 p^2 \nonumber \\
    &=& np( p(n-1) + 1 - np ) \nonumber \\
    &=& np( 1-p )
\end{eqnarray}

%%%%%%%%%%%%%%%%%%%%%%%%%%%%%%%%%%%%%%%%%%%%%%%%%%
\problem

\begin{eqnarray}
  \sum_{k=0}^{\infty} k p_k &=& 1p_1 + 2p_2 + 3p_3 +... \nonumber \\
                            &=& p_1 + p_2 + p_3 + ... + p_2 + p_3 + p_4 + ... + p_3 + p_4 + p_5 + ... \nonumber \\
                            &=& \sum_{k=0}^{\infty}(p_{k+1} + p_{k+2} + p_{k+3} + ...) \nonumber \\
                            &=& \sum_{k=0}^{\infty}t_k = E[X] = T(1) = P'(1)
\end{eqnarray}

%%%%%%%%%%%%%%%%%%%%%%%%%%%%%%%%%%%%%%%%%%%%%%%%%%
\problem

Each $X_i$ is an independent Bernoulli random variable. If we let $a_k = (a_0, a_1)$, then we can obtain the $pgf$ of the Bernoulli distribution.

\begin{eqnarray}
  A(s) &=& \sum_{k=0}^{\infty} a_k s^k \nonumber \\
       &=& a_0 + a_1 s \nonumber \\
       &=& q + ps
\end{eqnarray}

If $S$ is the sum of $n$ such random variables, then we can get the $pgf$ of $S$ using equation 1.7.
\begin{eqnarray}
  A_{X_1 + ... + X_n}(s) &=& \prod_{i=1}^{n}A_{X_i}(s) \nonumber \\
                         &=& \prod_{i=1}^{n}(q + ps) \nonumber \\
                         &=& (q + ps)^n
\end{eqnarray}

So the $pgf$ for $S$ is the $pgf$ for the binomial distribution.

%%%%%%%%%%%%%%%%%%%%%%%%%%%%%%%%%%%%%%%%%%%%%%%%%%
\problem

asdf

\section*{Problems}
\setcounter{prob_num}{1}

%%%%%%%%%%%%%%%%%%%%%%%%%%%%%%%%%%%%%%%%%%%%%%%%%%
\problem

Because $D(s) = (s - s_1)D^{*}(s)$, we can write $D'(s)$ as follows. \[ D'(s) = (s - s_1)'D^{*}(s) + (s-s_1)D^{*'}(s) \] If we assume that $s_1$ is the shared root, then we have the following \[ \rho_1 = \frac{(s_1 - s_1)N^{*}(s_1)}{D^{*}(s_1) + (s_1 - s_1)D^{*'}(s_1)} = \frac{0}{D^{*}(s_1)} = 0 \] The remainder of the $\rho_i$ values are unchanged. Consider, without loss of generality, $\rho_2$. \[ \rho_2 = \frac{N(s_2)}{D'(s_2)} = \frac{(s_2 - s_1)N^{*}(s_2)}{D^{*}(s_2) + (s_2 - s_1)D^{*'}(s_2)} = \frac{(s_2 - s_1)N^{*}(s_2)}{(s_2 - s_1)D^{*'}(s_2)} = \frac{N^{*}(s_2)}{D^{*'}(s_2)} \]

%%%%%%%%%%%%%%%%%%%%%%%%%%%%%%%%%%%%%%%%%%%%%%%%%%
\problem

Let us assume $m = 1$. Therefore $D(s)$ is a $0^{\text{th}}$-order polynomial (constant) with 1 root. If $D(s)$ is a constant and has a root, then that constant must be 0, so $D(s)$ is the null polynomial.

Now let us assume that conditions hold for $m = k$ (that is, a $(k-1)^{\text{th}}$-order polynomial with $k$ distinct roots must be the null polynomial). Let us then consider when $m = k+1$ (that is, a $k^{\text{th}}$-order polynomial with $k+1$ distinct roots). Let $r$ be one of these roots. We can write $D(s)$ like so. \[ D(r) - D(s) = (r - s)\left[ a_1 + a_2(r + s)+ ... +a_k(r^{k-1} + r^{k-2}s + ... + s^{k-1}) \right]  \] The bracketed component is a $(k-1)^{\text{th}}$-order polynomial with $k$ distinct roots, so by the induction hypothesis it is the null polynomial. Therefore $D(s)$ is also the null polynomial.

%%%%%%%%%%%%%%%%%%%%%%%%%%%%%%%%%%%%%%%%%%%%%%%%%%
\problem

Asdf.

%%%%%%%%%%%%%%%%%%%%%%%%%%%%%%%%%%%%%%%%%%%%%%%%%%
\problem


\end{document}
