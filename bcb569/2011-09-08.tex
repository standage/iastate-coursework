\documentclass[10pt]{article}
\usepackage[margin=1in]{geometry}
\usepackage[shortlabels]{enumitem}
%\setcounter{secnumdepth}{0}
\usepackage{amssymb,amsmath,amsthm}
\usepackage{graphicx}
\usepackage{caption}
\usepackage{fancyhdr, lastpage}
\pagestyle{fancy}
\fancyhf{}
\lhead{Daniel Standage}
\chead{BCB 569, 2:10pm T/Th}
\rhead{Lecture Notes: 8 Sep, 2011}
%\cfoot{Page \thepage{} of \protect\pageref*{LastPage}}
\usepackage{varioref}
\labelformat{equation}{(#1)}
\usepackage[colorlinks,linkcolor=blue]{hyperref}

\newenvironment{mitemize}
{
  \begin{itemize}
  \setlength{\itemsep}{1pt}
  \setlength{\parskip}{0pt}
  \setlength{\parsep}{0pt}}{\end{itemize}
}

\newenvironment{menumerate}
{
  \begin{enumerate}
  \setlength{\itemsep}{1pt}
  \setlength{\parskip}{0pt}
  \setlength{\parsep}{0pt}}{\end{enumerate}
}


\begin{document}

\section*{Conversion between Cartesian and internal coordinates}

\subsection*{Cartesian}
\begin{mitemize}
  \item $(x_1, y_1, z_1) \rightarrow$ atom 1
  \item ...
  \item ...
  \item $(x_n, y_n, z_n) \rightarrow$ atom n
  \item $n$ residues, $3n$ backbone atoms $\Rightarrow 3n \times 3 = 9n$ variables required to represent the structure
  \item finding a trajectory between two structures, linear interpolation (in Cartesian space) does not guarantee backbone structure is chemically sane
\end{mitemize}

\subsection*{Internal}
$(l_i, \theta_i, \phi_i)$

\begin{mitemize}
  \item \textit{cis}-configuration $\Rightarrow \phi = 0$
  \item \textit{trans}-configuration $\Rightarrow \phi = 180$
  \item $\phi = 180^{o} \Rightarrow \phi = -180^{o}$
  \item $3n - 1$ bond lenghts, $3n - 2$ bond angles ($\theta$), $3n-3$ torsional angles ($\phi$) $\Rightarrow 3n - 1 + 3n - 2 + 3n - 3 = 9n - 6$ variables required to represent the structure; however, this is an upper limit; often bond lengths and bond angles can be treated as constant
  \item finding trajectory between two structures, interpolation over internal coordinates guarantees backbone structure is chemically sane (does not guarantee that side chains won't collide though---this is still an open research problem)
\end{mitemize}

\section*{Empirical models and force fields}
Modeling behavior of a protein requires 1: a model of the atoms and 2: the interactions between them. Atoms are complex, but we can ignore most of this complexity and treat an atom as a point or a ball. However, there are some properties we want to model explicitly: the mass, the charge, and the radius. Parameters required to model an atom: $(m, q, r, x, y, z, V_x, V_y, V_z)$.

\end{document}