\documentclass[10pt]{article}
\usepackage[margin=1in]{geometry}
\usepackage[shortlabels]{enumitem}
%\setcounter{secnumdepth}{0}
\usepackage{amssymb,amsmath,amsthm}
\usepackage{graphicx}
\usepackage{caption}
\usepackage{fancyhdr, lastpage}
\pagestyle{fancy}
\fancyhf{}
\lhead{Daniel Standage}
\chead{BCB 569, 2:10pm T/Th}
\rhead{Lecture Notes: 25 Aug, 2011}
%\cfoot{Page \thepage{} of \protect\pageref*{LastPage}}
\usepackage{varioref}
\labelformat{equation}{(#1)}
\usepackage[colorlinks,linkcolor=blue]{hyperref}

\newenvironment{mitemize}
{
  \begin{itemize}
  \setlength{\itemsep}{1pt}
  \setlength{\parskip}{0pt}
  \setlength{\parsep}{0pt}}{\end{itemize}
}

\newenvironment{menumerate}
{
  \begin{enumerate}
  \setlength{\itemsep}{1pt}
  \setlength{\parskip}{0pt}
  \setlength{\parsep}{0pt}}{\end{enumerate}
}


\begin{document}

\section*{Paradigm shift in protein science}
  \subsection*{Central paradigm}
    \begin{mitemize}
      \item old idea: sequence determines structure, which determines function
      \item new idea: sequence determines structure \textit{which determines dynamics} which determines function
    \end{mitemize}
  
  \subsection*{Native state(s) of a protein}
    \begin{mitemize}
      \item old idea: a single, static structure (native state)
      \item new idea: a structural ensemble
      \item ubiquitin example: induced fit (different states caused by binding) vs conformational selection (protein exists in multiple forms, substrate binds the correct state)
    \end{mitemize}

\section*{Flow of genetic information}
  \begin{mitemize}
    \item DNA is a chain of 4 nucleotides
    \item proteins are chains of 20 amino acids
    \item dogma of molecular biology: DNA $\rightarrow$ RNA $\rightarrow$ protein
    \item genetic code: possible codons (triplets) of 4 nucleotides: $4^3 = 64$; single base changes usually result in same amino acid or an amino acid with similar physical properties
  \end{mitemize}
  
\section*{Protein structure}
  \begin{mitemize}
    \item protein is a change of amino acids
    \item amino acids share a backbone, distinguished by the side chains
    \item classified as hydrophobic, hydrophilic, amphipathic
    \item hydrophobic
    \begin{mitemize}
      \item engage in van der Waals interactions only; cannot form hydrogen bonds with water
      \item normally on interior of protein, packing against each other
      \item basis of \textbf{hydrophobic effect}
    \end{mitemize}
    \item hydrophilic: form hydrogen bonds with water
    \item amphipathic: can form hydrogen bonds, but have a large hydrophobic component
  \end{mitemize}

\section*{Protein flexibility}
  \begin{mitemize}
    \item bond length and bond angles are (nearly) fixed
    \item peptide group is planar
    \item flexibility comes from the torsional angles phi ($\phi$) and psi ($\psi$) (two degrees of freedom on the backbone per residue)
  \end{mitemize}
  \subsection*{Side chain flexibility}
    \begin{mitemize}
      \item backbone flexibility comes from $\phi$ and $\psi$
      \item do torsional rotations exist in side chains? how many degrees of freedom are there for each side chain?
      \item what is the effect of chain torsional rotation?
      \item effects of side chain rotation are local, so it is less constrained
    \end{mitemize}

\section*{Protein conformations}
  \begin{mitemize}
    \item a conformation is a 3D form that a protein can adapt without covalent bond breakage
    \item the set of all possible confirmations are is called the conformational space; this space is high-dimensional
    \item Ramachandan plot and steric constraints: brighter areas show preferred combinations of $\phi$ and $\psi$
  \end{mitemize}
  \subsection*{Folding and the native structure}
    \begin{mitemize}
      \item native state is 3D, compact, and lowest energy
      \item folding is the process by which a protein finds a single native state within a large conformational space
      \item Levinthal paradox; conformational space is huge, but protein folds almost instantaneously
    \end{mitemize}

\end{document}