\documentclass[10pt]{article}
\usepackage[margin=1in]{geometry}
\usepackage[shortlabels]{enumitem}
%\setcounter{secnumdepth}{0}
\usepackage{amssymb,amsmath,amsthm}
\usepackage{graphicx}
\usepackage{caption}
\usepackage{fancyhdr, lastpage}
\pagestyle{fancy}
\fancyhf{}
\lhead{Daniel Standage}
\chead{BCB 569, 2:10pm T/Th}
\rhead{Lecture Notes: 20 Sep, 2011}
%\cfoot{Page \thepage{} of \protect\pageref*{LastPage}}
\usepackage{varioref}
\labelformat{equation}{(#1)}
\usepackage[colorlinks,linkcolor=blue]{hyperref}

\newenvironment{mitemize}
{
  \begin{itemize}
  \setlength{\itemsep}{1pt}
  \setlength{\parskip}{0pt}
  \setlength{\parsep}{0pt}}{\end{itemize}
}

\newenvironment{menumerate}
{
  \begin{enumerate}
  \setlength{\itemsep}{1pt}
  \setlength{\parskip}{0pt}
  \setlength{\parsep}{0pt}}{\end{enumerate}
}


\begin{document}

\section*{Protein structure determination}

Three things can happen depending on the relationship between the size of the object you're observing and the wavelength $\lambda$ of the radiation you're using to observe it.
\begin{mitemize}
  \item if $\lambda$ is significantly smaller than the size of the object, you get reflection, imaging
  \item if $\lambda$ is significantly larger than the size of the object, you get no resolution
  \item if $\lambda$ is close to the size of the object, you get diffraction
\end{mitemize}

The take-away message about X-ray diffraction vs NMR is this.
\begin{mitemize}
  \item X-rays are the cheapest, easiest way to get \AA-resolution radiation in the lab. Also, X-ray crystallography can handle very large proteins. However, you have to first crystalize your protein, and this may take a very long time---if you can even do it at all.
  \item NMR requires quite a bit of work, but you're pretty much guaranteed to get the data you want when all is said and done. However, there is a size limitation to the proteins you can analyze with NMR.
\end{mitemize}

\end{document}