\documentclass[10pt]{article}
\usepackage[margin=1in]{geometry}
\usepackage[shortlabels]{enumitem}
%\setcounter{secnumdepth}{0}
\usepackage{amssymb,amsmath,amsthm}
\usepackage{graphicx}
\usepackage{caption}
\usepackage{fancyhdr, lastpage}
\pagestyle{fancy}
\fancyhf{}
\lhead{Daniel Standage}
\chead{BCB 569, 2:10pm T/Th}
\rhead{Lecture Notes: 1 Sep, 2011}
%\cfoot{Page \thepage{} of \protect\pageref*{LastPage}}
\usepackage{varioref}
\labelformat{equation}{(#1)}
\usepackage[colorlinks,linkcolor=blue]{hyperref}

\newenvironment{mitemize}
{
  \begin{itemize}
  \setlength{\itemsep}{1pt}
  \setlength{\parskip}{0pt}
  \setlength{\parsep}{0pt}}{\end{itemize}
}

\newenvironment{menumerate}
{
  \begin{enumerate}
  \setlength{\itemsep}{1pt}
  \setlength{\parskip}{0pt}
  \setlength{\parsep}{0pt}}{\end{enumerate}
}


\begin{document}

\section*{Side chain flexibility}
In physics, \textbf{degrees of freedom} refers to the number of variables required to describe the system. It is an analog to the same idea in statistics, but used for different things.

\begin{mitemize}
  \item A single particle in space has 3 degrees of freedom (x,y,z)
  \item A 3D object in space has 6 degrees of freedom (x,y,z,yaw,pitch,roll)
\end{mitemize}

In summary...
\begin{mitemize}
  \item for each residue we have two torsional degrees of freedom (DOF) in the main chain (phi $\phi$ and psi $\psi$)
  \item each residue also has 0-4 DOF in its side chain, which affects the local form of this side chain 
  \item therefore, a residue can have up to 6 DOF
\end{mitemize}

\section*{Why do proteins fold?}

\begin{mitemize}
  \item reduce energy levels
  \item polar side chains want to interact with the water and with each other
  \item reducing hydrophobic surface area: non-polar side chains want to stay away from the water and polar side chains and interact with each other
  \item condensed non-polar residues cause protein to compact and assume a globular shape
\end{mitemize}

\subsection*{Secondary structure}
Folded proteins form regular structure patterns (secondary structures) to increase hydrogen bonding and reduce steric clashing.
\begin{mitemize}
  \item $\alpha$ helices
  \begin{mitemize}
    \item most common
    \item they have \textit{handedness} (or chirality); most are right-handed (left-handed helices have steric constraints)
    \item can be amphipathic
    \item has periodicity: residues 3-4 spaces apart may have the same face
  \end{mitemize}
  \item $\beta$ strands/sheets
  \begin{mitemize}
    \item can be formed from both parallel and antiparallel peptide strands
    \item parallel sheets are always buried, but antiparallel sheets are more stable
  \end{mitemize}
  \item $\beta$ turns: smallest secondary structure, stabilized by hydrogen bonding
\end{mitemize}

\section*{Calculations}
\begin{mitemize}
  \item If we have Cartesian coordinates of each amino acid, can we calculate bond lengths, bond angles, and torsional angles?
  \item Inversely, if we have bond lengths, bond angles, and torsional angles, can we calculate Cartesian coordinates?
\end{mitemize}


\end{document}