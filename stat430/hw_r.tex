\documentclass[a4paper, 10pt]{article}
\usepackage[latin1]{inputenc}    % Accept european-encoded (latin1) characters.
\usepackage{a4wide}              % Wide paper
\usepackage[parfill]{parskip}		% skip a line instead of indenting new paragraphs
\usepackage{graphicx}   % For eps figures
\usepackage{epsfig}     % Alternative package
 
\usepackage{fancyhdr} 
\fancyhead[L]{\assignment }
\fancyhead[R]{\class \;- \instructor }
\renewcommand{\footrulewidth}{0.5pt} % Insert a line above the footer
\pagestyle{fancy}
\usepackage[hang,small,bf]{caption}
\usepackage{palatino}
\usepackage{amsmath}
\usepackage{amssymb}
\usepackage{enumerate}
 
% convenience commands
\renewcommand{\author}{Daniel Standage}
\newcommand{\class}{Stat 430}
\newcommand{\instructor}{Wei-chen Chen}
\newcommand{\assignment}{R Tutorial}
 
\newcounter{prob_num}
\setcounter{prob_num}{1}
% usage: \problem
\newcommand{\problem}{\vspace{20pt}\arabic{prob_num}.\stepcounter{prob_num}\par}
% usage: \head{name}{class}{assignment}
\newcommand{\head}{\begin{center}\begin{tabular*}{\linewidth}{l@{\extracolsep{\fill}}r}\author & \class\;- \instructor\\ \today & \assignment\end{tabular*}\end{center} \hfill }
% usage: \eqn{equation}{label}
\newcommand{\eqn}[2]{\begin{equation}#1\label{#2}\end{equation}}
 
% begin document
\begin{document}
 
\head
 
%%%%%%%%%%%%%%%%%%%%%%%%%%%%%%%%%%%%%%%%%%%%%%%%%%
\problem

The \textit{seq(n,k)} function returns a vector of all the integers between $n$ and $k$ (inclusive), while the \textit{rep(n,k)} returns a vector containing $k$ entries of the value $n$. The results are shown below.
\begin{verbatim}> seq(1,5)
[1] 1 2 3 4 5
> rep(1,5)
[1] 1 1 1 1 1
\end{verbatim}

%%%%%%%%%%%%%%%%%%%%%%%%%%%%%%%%%%%%%%%%%%%%%%%%%%
\problem

\begin{enumerate}[(a)]
\item
  \begin{itemize}
    \item The $\chi^2$ distribution uses the suffix \textit{chisq} (e.g., \textit{dchisq, qchisq, pchisq, rchisq})
    \item The $t$ distribution uses the suffix $t$ (e.g., \textit{dt, qt, pt, rt})
    \item The $F$ distribution uses the suffix $f$ (e.g., \textit{df, qf, pf, rf})
    \item The \textit{Binomial} distribution uses the suffix \textit{binom} (e.g., \textit{dbinom, qbinom, pbinom, rbinom})
    \item The \textit{Poisson} distribution uses the suffix \textit{pois} (e.g., \textit{dpois, qpois, ppois, rpois})
  \end{itemize}
\item These functions belong to the \textit{stats} library.
\end{enumerate}
 
%%%%%%%%%%%%%%%%%%%%%%%%%%%%%%%%%%%%%%%%%%%%%%%%%%
\problem

\begin{enumerate}[(a)]
\item These are not the same because the \textit{qnorm} function returns a quantile, not a probability.
\item These are the same because the \textit{pexp} function returns a probability.
\end{enumerate}

%%%%%%%%%%%%%%%%%%%%%%%%%%%%%%%%%%%%%%%%%%%%%%%%%%
\problem

\begin{enumerate}[(a)]
\item These functions belong to the \textit{graphics} library.
\item These functions write plots out to graphics files so that they can be viewed in other programs.
\end{enumerate}

%%%%%%%%%%%%%%%%%%%%%%%%%%%%%%%%%%%%%%%%%%%%%%%%%%
\problem

\begin{enumerate}[(a)]
\item These functions are for performing Wilcoxon tests, performing analysis of variance, and fitting linear models (respectively).
\item The \textit{lme} function belongs to the \textit{nlme} library.
\end{enumerate}
 
\end{document}