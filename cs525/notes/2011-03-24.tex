\documentclass[10pt]{article}
\usepackage[margin=1in]{geometry}
\usepackage[shortlabels]{enumitem}
%\setcounter{secnumdepth}{0}
\usepackage{amssymb,amsmath,amsthm}
\usepackage{graphicx}
\usepackage{caption}
\usepackage{fancyhdr, lastpage}
\pagestyle{fancy}
\fancyhf{}
\lhead{Daniel Standage}
\chead{CS 525, 11:00am T/Th}
\rhead{Lecture Notes: Mar 24, 2011}
%\cfoot{Page \thepage{} of \protect\pageref*{LastPage}}
\usepackage{varioref}
\labelformat{equation}{(#1)}
\usepackage[colorlinks,linkcolor=blue]{hyperref}

\newenvironment{mitemize}
{
  \begin{itemize}
  \setlength{\itemsep}{1pt}
  \setlength{\parskip}{0pt}
  \setlength{\parsep}{0pt}}{\end{itemize}
}

\newenvironment{menumerate}
{
  \begin{enumerate}
  \setlength{\itemsep}{1pt}
  \setlength{\parskip}{0pt}
  \setlength{\parsep}{0pt}}{\end{enumerate}
}


\begin{document}

\begin{verbatim}
Revised homework 6.

right shift; (mpi_send and mpi_recv)
              how can we get the most efficient right shift using non-blocking routines?
              It may use blocking routines
\end{verbatim}

\section*{Illustrate domain decomposition}
\begin{verbatim}
do iter = 1, niter
  do j = 1,n
    do i = 1, n
      B(i,j) = 0.25d0 [A(i-1,j) + A(i+1, j) + A(1, j-1)+ A(1, j+1)]
    enddo
  enddo
  do j=1,n
    do i = i,n
      A(i,j) = B(i,j) ! stride 1
    enddo
  enddo
enddo
\end{verbatim}

\subsection*{Parallelization: column blocks}

\end{document}