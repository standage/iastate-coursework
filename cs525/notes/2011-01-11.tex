\documentclass[10pt]{article}
\usepackage[margin=1in]{geometry}
\usepackage[shortlabels]{enumitem}
%\setcounter{secnumdepth}{0}
\usepackage{amssymb,amsmath,amsthm}
\usepackage{graphicx}
\usepackage{caption}
\usepackage{fancyhdr, lastpage}
\pagestyle{fancy}
\fancyhf{}
\lhead{Daniel Standage}
\chead{CS 525, 11:00am T/Th}
\rhead{Lecture Notes: January 11, 2011}
%\cfoot{Page \thepage{} of \protect\pageref*{LastPage}}
\usepackage{varioref}
\labelformat{equation}{(#1)}
\usepackage[colorlinks,linkcolor=blue]{hyperref}

\newenvironment{mitemize}
{
  \begin{itemize}
  \setlength{\itemsep}{1pt}
  \setlength{\parskip}{0pt}
  \setlength{\parsep}{0pt}}{\end{itemize}
}

\newenvironment{menumerate}
{
  \begin{enumerate}
  \setlength{\itemsep}{1pt}
  \setlength{\parskip}{0pt}
  \setlength{\parsep}{0pt}}{\end{enumerate}
}


\begin{document}

\section{Introduction}
  \subsection{Grading}
    \begin{mitemize}
      \item Homework $\approx \frac{1}{3}$ grade
      \item Final exam $\approx \frac{1}{3}$ grade
      \item Project $\approx \frac{1}{3}$ grade
    \end{mitemize}

  \subsection{Course Outline}
    \begin{mitemize}
      \item Intro to HPC
      \item Parallelization using MPI
      \item Program opt: serial, parallel
      \item Overview of using OpenMPI for shared memory paralellization
      \item Semester projects
    \end{mitemize}

  \subsection{Machines}
    \begin{mitemize}
      \item HPC class (here at ISU--shared memory not feasible)
      \item Ranger at TACC (Austin, Texas)
      \item Semester projects...you choose!
    \end{mitemize}


\section{What is HPC?}
The primary goal of HPC is speed, often achieved using multiple processors and/or cores

  \subsection{Examples}
  \begin{mitemize}
    \item Cummins Diesel
    \begin{mitemize}
      \item 11 processors, 256 GB RAM (this was huge back then)
      \item simulated entire physical process of running diesel engine
      \item memory to CPU access is much quicker than disk to CPU access
    \end{mitemize}
    \item Pratt and Whitney
    \begin{mitemize}
      \item Could not simulate entire jet engine at the same time, had to break down into stages
    \end{mitemize}
    \item Car crash testing: cannot compete in auto industry testing today without HPC
    \item Human genome mapping would not be possible without HPC
    \item Financial modeling
    \begin{mitemize}
      \item Bond firm gets rich with switch to HPC (estimates in 4 hours rather than 24)
      \item High-speed trading (kind of like gambling)
    \end{mitemize}
  \end{mitemize}

\section{Overview of HPC}
Parallelization can be accomplished with shared or distributed memory

\begin{mitemize}
  \item shared memory is easier to program
  \item however, shared memory paradigm does not scale well (to lots of processors) since the memory bus gets saturated
  \item Stanford (SGI) solution: common address space for all (distributed) memory
\end{mitemize}

  \subsection{Today's HPC machines}
  \begin{mitemize}
    \item Collection of shared memory compute nodes
    \item Until about 1990, most HPC used custom (expensive) hardware $ \Rightarrow $ low volume, harder to make a profit
    \item Now with large volume, processors are very cheap
    \item Today's HPC hardware is typically high volume commodity hardware used in smaller devices
    \item This is the idea behind GPUs--high volume, low cost; however, GPU is fast only for single precision arithmetic (most science applications need double precision)); lacked ECC (error correcting memory--necessary for scientific computing)
    \item Top 3 machines for Linpock Benchmark
    \begin{menumerate}
      \item GPU machine
      \item Jaguar at ORNL
      \item Chinese GPU machine
    \end{menumerate}
  \end{mitemize}
  

\end{document}