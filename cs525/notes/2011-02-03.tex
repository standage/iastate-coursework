\documentclass[10pt]{article}
\usepackage[margin=1in]{geometry}
\usepackage[shortlabels]{enumitem}
%\setcounter{secnumdepth}{0}
\usepackage{amssymb,amsmath,amsthm}
\usepackage{graphicx}
\usepackage{caption}
\usepackage{fancyhdr, lastpage}
\pagestyle{fancy}
\fancyhf{}
\lhead{Daniel Standage}
\chead{CS 525, 11:00am T/Th}
\rhead{Lecture Notes: Feb 3, 2011}
%\cfoot{Page \thepage{} of \protect\pageref*{LastPage}}
\usepackage{varioref}
\labelformat{equation}{(#1)}
\usepackage[colorlinks,linkcolor=blue]{hyperref}

\newenvironment{mitemize}
{
  \begin{itemize}
  \setlength{\itemsep}{1pt}
  \setlength{\parskip}{0pt}
  \setlength{\parsep}{0pt}}{\end{itemize}
}

\newenvironment{menumerate}
{
  \begin{enumerate}
  \setlength{\itemsep}{1pt}
  \setlength{\parskip}{0pt}
  \setlength{\parsep}{0pt}}{\end{enumerate}
}


\begin{document}

\subsection{\texttt{mpi\_recv}}
\begin{mitemize}
  \item Can use \texttt{mpi\_any\_source} and \texttt{mpi\_any\_tag} for the source and tag arguments (respectively); should use them when possible, but only when needed
  \item \texttt{status(mpi\_source) =} source of the message just received
  \item \texttt{status(mpi\_tag) =} tag of the message just received
  \item \texttt{status(mpi\_error) =} not useful
  \item \texttt{status(mpi\_status\_ignore)}; reduces overhead related to status tracking
\end{mitemize}


\section{Sending routines}

\subsection{Blocking sends}
\begin{mitemize}
  \item \texttt{mpi\_ssend}: synchronous send; hand-shake between sending and receiving processes, must complete before the processes proceed with execution;
  \item \texttt{mpi\_rsend}: send results immediately, requires a receive to be waiting; if receive is not waiting, then behavior is undefined; shouldn't use
  \item \texttt{mpi\_bsend}: buffered send; copies message to a buffer and execution continues; can help with load balancing; caveat is that the user must manage the buffer space; (see \texttt{mpi\_buffer\_attach} and \texttt{mpi\_buffer\_detach})
  \item \texttt{mpi\_send}: IBM made their implementation public a few years ago (if message was small enough and the number of MPI processes was not too large, message would be sent to a buffer on the receiving process; otherwise, used \texttt{mpi\_ssend}); notice that issues may not appear until large problems are run; debugging with \texttt{mpi\_ssend} is critical to avoiding such mistakes
\end{mitemize}

\subsection{Non-blocking routines}
The intent of non-blocking routines is to enable communication and computation to overlap and to avoid deadlocking

\subsubsection{Deadlocks (deadly embrace)}
Basically, you're waiting for something to happen that is never going to happen. You can avoid them by
\begin{mitemize}
  \item be careful how sends and receives are ordered
  \item use \texttt{mpi\_sendrecv} or \texttt{mpi\_sendrecv\_replace}
  \item use nonblocking routines (if you can)
  \item use buffered sends
\end{mitemize}

\end{document}