\documentclass[10pt]{article}
\usepackage[margin=1in]{geometry}
\usepackage[shortlabels]{enumitem}
%\setcounter{secnumdepth}{0}
\usepackage{amssymb,amsmath,amsthm}
\usepackage{graphicx}
\usepackage{caption}
\usepackage{fancyhdr, lastpage}
\pagestyle{fancy}
\fancyhf{}
\lhead{Daniel Standage}
\chead{CS 525, 11:00am T/Th}
\rhead{Lecture Notes: Feb 1, 2011}
%\cfoot{Page \thepage{} of \protect\pageref*{LastPage}}
\usepackage{varioref}
\labelformat{equation}{(#1)}
\usepackage[colorlinks,linkcolor=blue]{hyperref}

\newenvironment{mitemize}
{
  \begin{itemize}
  \setlength{\itemsep}{1pt}
  \setlength{\parskip}{0pt}
  \setlength{\parsep}{0pt}}{\end{itemize}
}

\newenvironment{menumerate}
{
  \begin{enumerate}
  \setlength{\itemsep}{1pt}
  \setlength{\parskip}{0pt}
  \setlength{\parsep}{0pt}}{\end{enumerate}
}


\begin{document}

Notes on homework 0
\begin{mitemize}
  \item Use exact notation from homework (dp, rank, comm, etc)
  \item Must run correctly for any $n$ and $p \geq 2$ 
\end{mitemize}

\section{MPI Overview}
\begin{menumerate}
  \item Point-to-point routines
  \begin{mitemize}
    \item blocking: \texttt{mpi\_send}, \texttt{mpi\_ssend}, \texttt{mpi\_recv}
    \item non-blocking: \texttt{mpi\_isend}, \texttt{mpi\_issend}, \texttt{mpi\_irecv}
  \end{mitemize}
  \item Collective routines: involves all processors in the \texttt{comm} (i.e. \texttt{mpi\_barrier}, \texttt{mpi\_bcast})
  \item MPI 1-sided routines (vs Cray's shmem (shared memory) routines; complicated but very fast)
  \item MPI I/O; quite useful, but complicated ($\approx$ 100 pages); Parallel I/O
  \item MPI process creation (not really useful to most people)
\end{menumerate}

\section{Point-to-Point Communication}
\begin{menumerate}
  \item blocking
  \begin{mitemize}
    \item \texttt{mpi\_ssend}
    \item \texttt{mpi\_rsend}
    \item \texttt{mpi\_bsend}
    \item \texttt{mpi\_send}
    \item \texttt{mpi\_recv}
    \item \texttt{mpi\_sendrecv}
    \item \texttt{mpi\_sendrecv\_replace}
  \end{mitemize}
  \item non-blocking
  \begin{mitemize}
    \item \texttt{mpi\_issend}
    \item \texttt{mpi\_irsend}
    \item \texttt{mpi\_ibsend}
    \item \texttt{mpi\_isend}
    \item \texttt{mpi\_irecv}
  \end{mitemize}
\end{menumerate}

\subsection{Difference between blocking and non-blocking routines}
When is the statement following the \texttt{send} call executed? For blocking routines, it waits until it is safe to modify the variable containing the data to be passed. Non-blocking routines do not offer this safety?

Why have non-blocking routines? It can hep avoid deadlocks and allow communication and execution to occur at the same time.

\subsection{Blocking sends}
\begin{mitemize}
  \item \texttt{mpi\_ssend}: synchronize send; slow for small messages, but performs fine with large messages; great for debugging
  \item \texttt{mpi\_rsend}: ready send; supposed to provide a very fast send; sends message immediately and the corresponding \texttt{recv} must be ``posted" (waiting for the message); if the \texttt{recv} is not posted the results are undefined
\end{mitemize}

\end{document}