\documentclass[10pt]{article}
\usepackage[margin=1in]{geometry}
\usepackage[shortlabels]{enumitem}
%\setcounter{secnumdepth}{0}
\usepackage{amssymb,amsmath,amsthm}
\usepackage{graphicx}
\usepackage{caption}
\usepackage{fancyhdr, lastpage}
\pagestyle{fancy}
\fancyhf{}
\lhead{Daniel Standage}
\chead{CS 525, 11:00am T/Th}
\rhead{Lecture Notes: January 18, 2011}
%\cfoot{Page \thepage{} of \protect\pageref*{LastPage}}
\usepackage{varioref}
\labelformat{equation}{(#1)}
\usepackage[colorlinks,linkcolor=blue]{hyperref}

\newenvironment{mitemize}
{
  \begin{itemize}
  \setlength{\itemsep}{1pt}
  \setlength{\parskip}{0pt}
  \setlength{\parsep}{0pt}}{\end{itemize}
}

\newenvironment{menumerate}
{
  \begin{enumerate}
  \setlength{\itemsep}{1pt}
  \setlength{\parskip}{0pt}
  \setlength{\parsep}{0pt}}{\end{enumerate}
}


\begin{document}

\section{Performance Evaluation}
\begin{mitemize}
  \item flops is a standard to measure of performance; number of floating-point operations per second (flops); megaflops = mflops = 1000 flops
  \begin{mitemize}
    \item gigaflops
    \item teraflops
    \item petaflops (range of current top 500 supercomputers)
    \item exaflops
    \item numbers reported are usually peak theoretical flops or tests on Linpack benchmark (good implementations achieve approx. 80\% peak theoretical)
    \item gives no evaluation of interconnect performance; example NASA-Ames 1985, Intel processors for CFD (computational fluid dynamics) application, 80\% of the processors were idle waiting for data
  \end{mitemize}
  \item MIPS = millions of instructions per second (non-scientific computing, banking, business, etc)
  \item Perfect club (1990s), University of Illinois Urbana-Champagne, I/O benchmark that failed around 1995
  \item HPC challenge benchmarks
  \item Intel Paragon
  \item SPEC benchmarks; unfortunately the code is proprietary
\end{mitemize}

\subsection{Some numbers as of May 2010}
\begin{mitemize}
  \item Intel Westmere 6 core proc
  \begin{mitemize}
    \item 2.66 GHz - \$1200
    \item 2.80 GHz - \$1500 (25\% increase cost, 2\% faster)
    \item 2.93 GHz - \$1750 (46\% increase cost, 10\% faster)
    \item 3.33 GHz - \$2000 (67\% increase cost, 25\% faster)
  \end{mitemize}
  \item AMD Opteron 6000 8 core proc
  \begin{mitemize}
    \item 2.3 GHz - \$600
    \item 2.4 GHz - \$850 (42\% increase cost, 4\% faster)
  \end{mitemize}
  \item AMD Opteron 6000 12 core proc
  \begin{mitemize}
    \item 2.3 GHz - \$1150
    \item 2.4 GHz - \$1600
  \end{mitemize}
  \item NVIDIA S2070 (GPU)
  \begin{mitemize}
    \item 2.3 GHz - \$3450 (programming model is not clear)
  \end{mitemize}
\end{mitemize}

\subsection{Amdahl's Law}
\begin{mitemize}
  \item From 1970s: If 10\% of a program is serial, and the other 90\% can be perfectly parallelized, and the program runs in 100 seconds, then the runtime is \[ \text{Time} = 10 + \frac{90}{p} \] So no matter how much we increase $p$, the best runtime is still 10 seconds.
  \item From 1980s: For most applications, the percentage of time spent in serial portion $T_S \rightarrow 0$ as the problem size increases. 
\end{mitemize}

\end{document}