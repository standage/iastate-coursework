\documentclass[10pt]{article}
\usepackage[margin=1in]{geometry}
\usepackage[shortlabels]{enumitem}
%\setcounter{secnumdepth}{0}
\usepackage{amssymb,amsmath,amsthm}
\usepackage{graphicx}
\usepackage{caption}
\usepackage{fancyhdr, lastpage}
\pagestyle{fancy}
\fancyhf{}
\lhead{Daniel Standage}
\chead{CS 525, 11:00am T/Th}
\rhead{Lecture Notes: January 25, 2011}
%\cfoot{Page \thepage{} of \protect\pageref*{LastPage}}
\usepackage{varioref}
\labelformat{equation}{(#1)}
\usepackage[colorlinks,linkcolor=blue]{hyperref}

\newenvironment{mitemize}
{
  \begin{itemize}
  \setlength{\itemsep}{1pt}
  \setlength{\parskip}{0pt}
  \setlength{\parsep}{0pt}}{\end{itemize}
}

\newenvironment{menumerate}
{
  \begin{enumerate}
  \setlength{\itemsep}{1pt}
  \setlength{\parskip}{0pt}
  \setlength{\parsep}{0pt}}{\end{enumerate}
}


\begin{document}

Postpone Intro to HPC for now

\section{History of MPI}
\begin{mitemize}
  \item 1980 - 1995
  \begin{menumerate}
    \item every vendor had their own message passing libraries
    \item ORNL created PVM (parallel virtual machine); it was free; used master/slave programming model
    \item Europe message passing library standard PAMACS; wanted a world standard, in 1995 led to MPI; MPI uses the SPMD programming paradigm, i.e., a single program runs on all processors
  \end{menumerate}
  \item 1988 - 1992: intense research effort to automatically parallelize serial programs
\end{mitemize}

\section{Overview of MPI}

A MPI communicator defines a group of processes to be used when running an MPI program. MPI has a predefined communicator named \texttt{mpi\_comm\_world} which consists of a single group of processes: $1,2,...,p-1$. Each process is assigned a rank. MPI groups can be restricted to subgroups and one can define \textit{intercommunicators} and \textit{intracommunicators} between subgroups. Most of the class, we will only use \texttt{mpi\_comm\_world}.

All MPI programs must contain the following.
\begin{menumerate}
  \item Call \texttt{mpi\_init(ierror)}: initializes the MPI environment (\texttt{ierror} is an int)
  \item Call \texttt{mpi\_comm\_size(comm, p, ierror)}: $p$ then is the number of MPI processes
  \item Call \texttt{mpi\_comm\_rank(comm, rank, ierror)}: rank of the executing processor
  \item Call \texttt{mpi\_finalize(ierror)}: clean up environment
\end{menumerate}

\end{document}