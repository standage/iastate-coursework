\documentclass[10pt]{article}
\usepackage[margin=1in]{geometry}
\usepackage[shortlabels]{enumitem}
%\setcounter{secnumdepth}{0}
\usepackage{amssymb,amsmath,amsthm}
\usepackage{graphicx}
\usepackage{caption}
\usepackage{fancyhdr, lastpage}
\usepackage{mhchem}
\pagestyle{fancy}
\fancyhf{}
\lhead{Daniel Standage}
\chead{BCB 570, 3:10pm MWF}
\rhead{Lecture Notes: 29 Feb, 2012}
%\cfoot{Page \thepage{} of \protect\pageref*{LastPage}}
\usepackage{varioref}
\labelformat{equation}{(#1)}
\usepackage[colorlinks,linkcolor=blue]{hyperref}

\newenvironment{mitemize}
{
  \begin{itemize}
  \setlength{\itemsep}{1pt}
  \setlength{\parskip}{0pt}
  \setlength{\parsep}{0pt}}{\end{itemize}
}

\newenvironment{menumerate}
{
  \begin{enumerate}
  \setlength{\itemsep}{1pt}
  \setlength{\parskip}{0pt}
  \setlength{\parsep}{0pt}}{\end{enumerate}
}


\begin{document}

\section*{Probabilistic boolean networks}

How does uncertainty propagate through a boolean network?

Consider a boolean network $G(V,F)$ containing $n$ genes $x_1, x_2, ..., x_n$, and an initial joint probability distribution \[ p(x), x = (x_1 ... x_n \in \{ 0,1 \}^n \]

\[ Pr[ f_1(x) = i_1, f_2(x) = i_2, ..., f_n(x) = i_n ] \]

\[ f_1(x) = i_1, f_2(x) = i_2, ..., f_n(x) = i_n ] \]
\[ = \bigcup_{k \leq n} O \times O ... x\{ f_k(x) = i_k \} \times O .. \times O \]

\[ \sum_{x \in A} Pr(x) \]
\[ A = \{ x \in {0,1}^n | f_k(x) = i_k \} \]

The sum implicitly defines an iterative map \[ p^{(t + 1)} = \psi(p^{(t)}) \]
One can show that if you write $p^{(t)} = (p_1^{(t)} p_2^{(t)} ... p_n^{(t)})$ then
\[ p^{(t+1)} = p^{(t)} P \] (What is $P$ equal to?)


\section*{The k-means algorithm}

We assume that we have $n$ data points $\{ a_j \}_{j=1}^{n} \in R^{m}$. Let \[ \Pi = \{\pi_i\}{i=1}^{k} \] denote a partition of $A$.
\[ \pi_j = \{ v | a_v \text{belongs to the jth cluster} \} \]

Let the "mean" (centroid) of the $j^{\text{th}}$ cluster be \[ m_j = \frac{1}{n_j}\{ \sum_{v \in \pi_j} a_v \} \]
where $n_j$ is the number of elements of $\pi_j$.

The "tightness" (coherence) of the cluster $\pi_j$ is defined as \[ q_j = \sum_{v \in \pi_j} || a_v - m_j ||^2 \]

The quality of clustering can be measured as the overall coherence of all clusters. \[ Q(\Pi) = \sum_{j=1}^{k} q_j \]



\end{document}
