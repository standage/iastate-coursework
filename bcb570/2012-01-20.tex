\documentclass[10pt]{article}
\usepackage[margin=1in]{geometry}
\usepackage[shortlabels]{enumitem}
%\setcounter{secnumdepth}{0}
\usepackage{amssymb,amsmath,amsthm}
\usepackage{graphicx}
\usepackage{caption}
\usepackage{fancyhdr, lastpage}
\usepackage{mhchem}
\pagestyle{fancy}
\fancyhf{}
\lhead{Daniel Standage}
\chead{BCB 570, 3:10pm MWF}
\rhead{Lecture Notes: 20 Jan, 2012}
%\cfoot{Page \thepage{} of \protect\pageref*{LastPage}}
\usepackage{varioref}
\labelformat{equation}{(#1)}
\usepackage[colorlinks,linkcolor=blue]{hyperref}

\newenvironment{mitemize}
{
  \begin{itemize}
  \setlength{\itemsep}{1pt}
  \setlength{\parskip}{0pt}
  \setlength{\parsep}{0pt}}{\end{itemize}
}

\newenvironment{menumerate}
{
  \begin{enumerate}
  \setlength{\itemsep}{1pt}
  \setlength{\parskip}{0pt}
  \setlength{\parsep}{0pt}}{\end{enumerate}
}


\begin{document}

\section*{Gillespie Method}

\subsection*{Nearest neighbor: random walk on a lattice}
We assume that the transition probability from site $i$ to site $j$ is given by the Poisson process  \[  W_{ij}\Delta t\]

Master equation: We are interested in the probability $p(x, t)$ of the particle being at site $x$ at time $t$.

\[ p(x, t + \Delta t) = p(x, t)(1 - W_{x,(x+1)}\Delta t - W_{x,(x-1)} \Delta t) + P(x-1, t)w_{x-1,x}\Delta t + P(x+1, x)W_{x+1,x}\Delta t \]
\[ P(x, t+\Delta t) - P(x,t) = -P(x,t)W_{x,x+1}\Delta t - P(x,t)W_{x,x-1}\Delta t + P(x+1, t)W_{x+1, x}\Delta t + P(x-1, t)W_{x-1,x}\Delta t \]
\[ \frac{P(x, t+\Delta t)-P(x,t)}{\Delta t} = -\sum_{k}P(x,t)W_{x,k} + \sum_{k}P(k,t)W_{k,x} \]

If we let $\Delta t \rightarrow 0$, we have 
\[ \frac{d}{dt}P(x,t) = \sum_{k}(P(k,x)W_{k,x} - P(x,t)W_{x,k}) \]
This is what is referred to as the Master equation.

\subsection*{Stochastic chemical kinetics}
In deterministic chemical kinetics we represent the state of te system by the concentrations of the various molecular species involved.
In stochastic chemical kinetics, the state of the system is represented by a vector \[ X(t) = [x_1(t) x_2(t) ... x_3(t)]^T \]

\[ A + B \rightarrow C + D \]
\[ X(t) = [A(t) B(t) C(t) D(t)]^T \]
\[ [11 20 0 0]^T \rightarrow [10 19 1 1]^T \rightarrow [9 18 2 2]^T \]


\end{document}
