\documentclass[10pt]{article}
\usepackage[margin=1in]{geometry}
\usepackage[shortlabels]{enumitem}
%\setcounter{secnumdepth}{0}
\usepackage{amssymb,amsmath,amsthm}
\usepackage{graphicx}
\usepackage{caption}
\usepackage{fancyhdr, lastpage}
\usepackage{mhchem}
\pagestyle{fancy}
\fancyhf{}
\lhead{Daniel Standage}
\chead{BCB 570, 3:10pm MWF}
\rhead{Lecture Notes: 11 Jan, 2012}
%\cfoot{Page \thepage{} of \protect\pageref*{LastPage}}
\usepackage{varioref}
\labelformat{equation}{(#1)}
\usepackage[colorlinks,linkcolor=blue]{hyperref}

\newenvironment{mitemize}
{
  \begin{itemize}
  \setlength{\itemsep}{1pt}
  \setlength{\parskip}{0pt}
  \setlength{\parsep}{0pt}}{\end{itemize}
}

\newenvironment{menumerate}
{
  \begin{enumerate}
  \setlength{\itemsep}{1pt}
  \setlength{\parskip}{0pt}
  \setlength{\parsep}{0pt}}{\end{enumerate}
}


\begin{document}

\section*{Law of Mass Action}
The Law of Mass Action states that the rate of product accumulation is proportional to the concentration of the reactants.
Consider again the following reaction. \[ A + B \xrightarrow{k} C \]
The LOMA the provides the following equation. \[ \frac{d}{dt}[C] = k[A][B] \]

\section*{Steady-state analysis}
Steady-state analysis is helpful only for small systems.
However, discussing the related theory will be helpful for analyses later on.

Recall the following two-way reaction. \[ \ce{A + B <=>[k^{+}][k^{-}] C} \]
At equilibrium, the concentration of reactants and products do not change.
In mathematical terms, the derivatives (velocity) of the concentrations is 0. \[ \frac{d}{dt}[A] = 0 \] %\begin{equation} \end{equation}
Hence, at equilibrium we can solve for $[C]$ as follows.
\begin{align*}
       \frac{d}{dt}[A] &= 0                         \\
k^{-}[C] - k^{+}[A][B] &= 0                         \\
              k^{-}[C] &= k^{+}[A][B]               \\
                   [C] &= \frac{k^{+}}{k^{-}}[A][B]
\end{align*}

If we assume that the reaction is the only reaction in the system, we have
\begin{align*}
                      [A] + [C] &= A_0 = [A]\mid_{t=0}               \\
                            [C] &= \frac{k^{+}}{k^{-}}(A_0 - [C])[B] \\
[C] + \frac{k^{+}}{k^{-}}[C][B] &= \frac{k^{+}}{k^{-}}A_0[B]         \\
[C](1 + \frac{k^{+}}{k^{-}}[B]) &= \frac{k^{+}}{k^{-}}A_0[B]         \\
                            [C] &= \frac{\frac{k^{+}}{k^{-}}A_0[B]}{1+\frac{k^{+}}{k^{-}}[B]} = \frac{A_0[B]}{\frac{k^{-}}{k^{+}} + [B]}
\end{align*}

\subsection*{Remark}
The quantity $\frac{k^{-}}{k^{+}} = K_0$ is called the dissociation constant, and its inverse $\frac{k^{+}}{k^{-}}$ is called the equilibrium constant.


\section*{Enzyme kinetics}
How do we model enzymatic reactions? This section will use the following abbreviations: $S$ for substrate, $E$ for enzyme, $P$ for product, $V$ for velocity of reaction, and $C$ for complex (substrate + enzyme).

\subsection*{First attempt}
\[ S + E \xrightarrow{k} P + E \] with the solution \[ \frac{d}{dt}[P] = k[S][E] \]
The problem with this approach is that according to the Law of Mass Action, $V$ should be linear in terms of $[S]$, whereas empirical evidence showed a logarithmic relationship.

\subsection*{Second attempt: Michaelis-Menton}
Michaelis and Menton added an additional step in the reaction with a chemical intermediate: a complex of the substrate and enzyme. \[ \ce{E + S <=>[\ce{k_1^{+}}][k_1^{-}] C \xrightarrow{k_2} P + E} \]
Let $x = [X]$ for all of the symbols defined in this section.

\begin{align*}
\frac{ds}{dt} &= k_1^{-}c - k_1^{+}es \\
\frac{de}{dt} &= k_1^{-}c + k_2c - k_1^{+}es = c(k_1^{-} + k_2)-k_1^{+}es \\
\frac{dc}{dt} &= k_1^{+}es - k_1^{-}c - k_2c = k_1{+}es - c(k_1^{-} + k_2)
\end{align*}

We see that $\frac{de}{dt}$ and $\frac{dc}{dt}$ have opposite and equal values, so $\frac{de}{dt} + \frac{dc}{dt} = 0$.
This makes chemical sense, since in a closed system, there is a constant amount of enzyme, and that enzyme is either bound or in complex with the substrate.

From this, we have $\frac{d}{dt}(e+c)=0$, which implies that the quantity $e+c$ is a constant value (derivative of a constant is 0).
Let $e+c=e_0$.
From this relationship, we have $e = e_0 - c$, which allows us to remove a term from our system of equations.
This concept will be very helpful in the future as we work with very large interaction networks.

We are interested in computing $V = \frac{dp}{dt} = k_2c$ in terms of $s$.
Michaelis and Menton introduced the equilibrium approximation when saturated with substrate: $\frac{ds}{dt} \approx 0$. From this we derive the following.
\begin{align*}
0 = \frac{ds}{dt} &= k_1^{-}c - k_1^{+}es \\
        k_1^{+}es &= k_1^{-}c \\
                c &= \frac{k_1^{+}}{k_1^{-}}es
\end{align*}

We know that $e + c = e_0 \rightarrow e = e_0 + c$, so we derive the following.
\begin{align*}
c &= \frac{k_1^{+}}{k_1^{-}}es \\
c &= \frac{k_1^{+}}{k_1^{-}}(e_0 - c)s \\
c &= \frac{e_0s}{K_0 + s}
\end{align*}

Therefore, we have \[ V = \frac{dp}{dt} = k_2c = \frac{k_2e_0s}{K_0 + s} = \frac{V_{max} \cdot s}{K_0 + s} \] which fits empirical data very well.

\end{document}
