\documentclass[10pt]{article}
\usepackage[margin=1in]{geometry}
\usepackage[shortlabels]{enumitem}
%\setcounter{secnumdepth}{0}
\usepackage{amssymb,amsmath,amsthm}
\usepackage{graphicx}
\usepackage{caption}
\usepackage{fancyhdr, lastpage}
\usepackage{mhchem}
\pagestyle{fancy}
\fancyhf{}
\lhead{Daniel Standage}
\chead{BCB 570, 3:10pm MWF}
\rhead{Lecture Notes: 1 Feb, 2012}
%\cfoot{Page \thepage{} of \protect\pageref*{LastPage}}
\usepackage{varioref}
\labelformat{equation}{(#1)}
\usepackage[colorlinks,linkcolor=blue]{hyperref}

\newenvironment{mitemize}
{
  \begin{itemize}
  \setlength{\itemsep}{1pt}
  \setlength{\parskip}{0pt}
  \setlength{\parsep}{0pt}}{\end{itemize}
}

\newenvironment{menumerate}
{
  \begin{enumerate}
  \setlength{\itemsep}{1pt}
  \setlength{\parskip}{0pt}
  \setlength{\parsep}{0pt}}{\end{enumerate}
}


\begin{document}

\section*{Stoichiometric matrix}

The stoichiometric matrix corresponds to adjacency matrix (graph theory). All kinetic laws can be expressed as \[ \frac{dx}{dt} = Sv \]
We have: $m$ reactions and $n$ molecular species

\[ x = (x_1 x_2 ... x_n)^T \in R^n \]
\[ v = (v_1 v_2 ... v_n)^T \in R^m \]

typical example: $A + B \rightarrow C, C \rightarrow A + B$

\[ \frac{d}{dt}[A] = k_2[C] - k_1[A][B] \]
\[ \frac{d}{dt}[B] = k_2[C] - k_1[A][B] \]
\[ \frac{d}{dt}[C] = k_1[A][B] - k_2[C] \]


\section*{Singular value decomposition}
Let $A$ be an mxn matrix.
Then, there exist orthogonal matrices $U$ and $V$ and a matrix $\Sigma$ such that $A = U \Sigma V^T$

\section*{Steady states}
A steady state is a state $x$ of the syste for which \[ \frac{dx}{dt} = S v = 0 \]
Solving for $v = (V_1 V_2 ... V_m)^T$ will give us the possible reaction rates at equilibrium

\section*{Matlab code}
\begin{verbatim}
% Solving linear systems in Matlab
A = [1 2, 0 1]
Y = [1, 2]

% If we want to solve the equation Ax = y, we do...
x=A\y

S = [-1 1 -1, 1 -1 0, 0 0 1]
v = S\zeros(3,1) % uh oh...singular matrix
null(S)
\end{verbatim}

We want to describe all possible solutions of $S v = 0$ (equation *). It turns out that ever solution $v$ of * can be written as \[ v = c_1 v_1 + c_2 v_2 + ... + c_k v_k \]
where $v_1 ... v_k$ are some special solutions of *.

\begin{verbatim}
S = [-1 1 0 0 -1 1, 0 -1 1 0 0 0, 0 0 -1 1 -1 1]
A = S'
null(A)
\end{verbatim}


\section*{Conservation relations}
\[ \frac{dx}{dt} = S v \] S is nxm

Consider a vector $x$ such that $x^T S = 0$

\[ y^T \frac{dx}{dt} = y^T(S v) \rightarrow \frac{dy^T x}{dt} = y^T S_v = 0 \rightarrow \frac{d}{dt}y^T x = 0 \]
This implies that $y^T x$ is constant. $y^T x = [y_1 ... y_n] [x_1 ... x_n]^T = y_1 x_1 + y_2 x_2 + ... + y_n x_n = $ constant

\subsection*{Example}
\begin{verbatim}
glucose --(ATP -> ADP)--> gluc_6P  <----> fruc_6P --(ATP -> ADP)--> fruc-1,6P

            S_1                      S_2            S_3                         S_4
                    S_5     S_6
                        V_1                   V_2                V_3


S = |  -1  0  0  |
    |   1 -1  0  |
    |   0  1 -1  |
    |   0  0  1  |
    |  -1  0 -1  |
    |   1  0  1  |
    
Matlab code starts here
----------------------------

A = [-1 1 0 0 -1 1, 0 -1 1 0 0 0, 0 0 -1 1 -1 1]
null(A)
% if you scale, you get this vector
[ 2 1 1 0 0 1, 0 0 0 0 1 1, 1 1 1 1 0 0 ]

% first column y_1, second y_2, third y_3
% additional conservation relation
% -y_1 + 3*y_2 + 2*y_3 = [0 1 1 2 3 2]^T
\end{verbatim}

\end{document}
