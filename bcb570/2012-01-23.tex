\documentclass[10pt]{article}
\usepackage[margin=1in]{geometry}
\usepackage[shortlabels]{enumitem}
%\setcounter{secnumdepth}{0}
\usepackage{amssymb,amsmath,amsthm}
\usepackage{graphicx}
\usepackage{caption}
\usepackage{fancyhdr, lastpage}
\usepackage{mhchem}
\pagestyle{fancy}
\fancyhf{}
\lhead{Daniel Standage}
\chead{BCB 570, 3:10pm MWF}
\rhead{Lecture Notes: 23 Jan, 2012}
%\cfoot{Page \thepage{} of \protect\pageref*{LastPage}}
\usepackage{varioref}
\labelformat{equation}{(#1)}
\usepackage[colorlinks,linkcolor=blue]{hyperref}

\newenvironment{mitemize}
{
  \begin{itemize}
  \setlength{\itemsep}{1pt}
  \setlength{\parskip}{0pt}
  \setlength{\parsep}{0pt}}{\end{itemize}
}

\newenvironment{menumerate}
{
  \begin{enumerate}
  \setlength{\itemsep}{1pt}
  \setlength{\parskip}{0pt}
  \setlength{\parsep}{0pt}}{\end{enumerate}
}


\begin{document}

\section*{Stochastic chemical kinetics}

The general Master Equation is given as follows. \[ \frac{d}{dt}p(x, t) = \sum_y\left[ W_{yx}p(x,t) - W_{x,y}p(x,t) \right] \]

We represented the state of the chemical system with the vector $X(t) = [X_1(t), X_2(t), ..., X_n(t)]^T$ where $X_i(t)$ represents the number of molecules of substance $i$.
We can do a random walk over a lattice, where each point is represented by an $n$-dimensional vector.

\subsection*{Michaelis-Menton}
We can revisit Michaelis-Menton now with the Master Equation notation. \[ \ce{S_1 + S_2 <=> S_3 \rightarrow S_4 + S_2} \]
Here, $S_1$ is substrate, $S_2$ is enzyme, $S_3$ is substrate/enzyme complex, and $S_4$ is product.

\begin{menumerate}
  \item $S_1 + S_2 \xrightarrow{C_1} S_3$
  \item $S_3 \rightarrow{C_2} S_1 + S_2$
  \item $S_3 \rightarrow{C_3} S_4 + S_2$
\end{menumerate}

State vector:  $S(t) = [S_1(t), S_2(t), S_3(t), S_4(t)]^t$.

With every part of the reaction (listed 1-3 above), we associate a stochiometric vector.

\begin{menumerate}
 \item $V_1 = [-1, -1, 1, 0]^T$
 \item $V_2 = [1, 1, -1, 0]^T$
 \item $V_3 = [0, 1, -1, 1]^T$
\end{menumerate}

Every time reaction $j$ occurs, the state of the system changes from state $S(t)$ to $S(t)+V_j$

What are the transition probabilities? The probability for a $V_j$ transition is given by $a_j(S(t))\Delta t$. How are the $a_j$'s computed? \[ a_1 = c_1 s_1 s_2 \] \[ a_3 = c_3 s_3 \]

\[ p(S, t + \Delta t) = p(s,t)\left[ 1 - \sum_{j=1}^{M} a_j(S(t))\Delta t \right] + \sum_{j=1}^{M}p(S-V_j,t)a_j(S(t))\Delta t \]
\[ \frac{d}{dt}p(s,t) = \sum_{j=1}^{M}p(S-V_j, t)a_j(S(t))-\sum_{j=1}^{M}p(s,t)a_j(S(t)) \]

Question: how are kinetic rates related to the propensities $a_j$?

\[ \frac{d}{dt}[S_3] = k_1[A][B]M\text{sec}^{-1} \]

Units: $k: M^{-1}\text{sec}^{-1}$, $a_j: \text{sec}^{-1}$, $c_j:$ (\# molecules)$^{-2}\text{sec}^{-1}$

\[ k_1 \sim M^{-1}\text{sec}^{-1} \sim \frac{L}{mole}\cdot \frac{1}{\text{sec}} \sim \frac{L}{N_A}\cdot\frac{1}{\text{sec}}\left( \frac{1}{L}\cdot\frac{1}{N_A} \right) \]

In general, $c_1 = \frac{k_1}{N_A \cdot \text{volume}}$

\end{document}
