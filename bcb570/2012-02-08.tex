\documentclass[10pt]{article}
\usepackage[margin=1in]{geometry}
\usepackage[shortlabels]{enumitem}
%\setcounter{secnumdepth}{0}
\usepackage{amssymb,amsmath,amsthm}
\usepackage{graphicx}
\usepackage{caption}
\usepackage{fancyhdr, lastpage}
\usepackage{mhchem}
\pagestyle{fancy}
\fancyhf{}
\lhead{Daniel Standage}
\chead{BCB 570, 3:10pm MWF}
\rhead{Lecture Notes: 8 Feb, 2012}
%\cfoot{Page \thepage{} of \protect\pageref*{LastPage}}
\usepackage{varioref}
\labelformat{equation}{(#1)}
\usepackage[colorlinks,linkcolor=blue]{hyperref}

\newenvironment{mitemize}
{
  \begin{itemize}
  \setlength{\itemsep}{1pt}
  \setlength{\parskip}{0pt}
  \setlength{\parsep}{0pt}}{\end{itemize}
}

\newenvironment{menumerate}
{
  \begin{enumerate}
  \setlength{\itemsep}{1pt}
  \setlength{\parskip}{0pt}
  \setlength{\parsep}{0pt}}{\end{enumerate}
}


\begin{document}

\section*{Applications of elementary flux modes, extreme pathways}
\begin{menumerate}
  \item engineer a network; e.g., optimize the production of a compound
  \item ``attack" a network
\end{menumerate}

Equilibrium fluxes can be expressed as \[ v = c_1 v_1 + c_2 v_2 + ... + c_r v_r \] where $S v_k = 0$ (we'll call this Eqn 1).

We start with the equation $S v = 0$. We decompose $v$ as \[ \begin{bmatrix} v_{rev} \\ v_{irr} \end{bmatrix} \] where the two subvectors correspond to the reversible and irreversible reactions in the network. We are interested in solutions to Eqn 1 that satisfy \[ v_{irr} \geq 0 \] That is, we constrain solutions so that coefficients associated with irreversible reactions are positive. Furthermore, we focus on solutions of Eqn 1 of the form \[ v = \sum_{k \in rev} c_k v_k + \sum_{k \in irr} \lambda_k v_k \] where $\lambda_k \geq 0$ and $c_k \in \mathbb{R}$.

A \emph{flux mode} is all vectors of the form \[ \lambda v^* \] where $v^*$ satisfies the 3 conditions just described.
\begin{menumerate}
  \item $S v^* = 0$
  \item $v^*_{irr} \geq 0$
  \item $\lambda > 0$
\end{menumerate}

An \emph{elementary flux mode} is a flux mode that cannot be further decomposed as a sum of other flux modes (not a proper basis since they are not necessarily linearly independent).

An \emph{extreme pathway} is an elementary flux mode where all reversible reactions are decomposed to two irreversible reactions, increasing the dimensionality of the solution space and constraining it to the non-negative space.

\end{document}
