\documentclass[10pt]{article}
\usepackage[margin=1in]{geometry}
\usepackage[shortlabels]{enumitem}
%\setcounter{secnumdepth}{0}
\usepackage{amssymb,amsmath,amsthm}
\usepackage{graphicx}
\usepackage{caption}
\usepackage{fancyhdr, lastpage}
\usepackage{mhchem}
\pagestyle{fancy}
\fancyhf{}
\lhead{Daniel Standage}
\chead{BCB 570, 3:10pm MWF}
\rhead{Lecture Notes: 13 Jan, 2012}
%\cfoot{Page \thepage{} of \protect\pageref*{LastPage}}
\usepackage{varioref}
\labelformat{equation}{(#1)}
\usepackage[colorlinks,linkcolor=blue]{hyperref}

\newenvironment{mitemize}
{
  \begin{itemize}
  \setlength{\itemsep}{1pt}
  \setlength{\parskip}{0pt}
  \setlength{\parsep}{0pt}}{\end{itemize}
}

\newenvironment{menumerate}
{
  \begin{enumerate}
  \setlength{\itemsep}{1pt}
  \setlength{\parskip}{0pt}
  \setlength{\parsep}{0pt}}{\end{enumerate}
}


\begin{document}

\section*{Enzyme kinetics}
In our previous discussions, we determined that (under the Law of Mass Action) the model \[ E + S \rightarrow P + E \] does not fit empirical data very well. 
Michaelis and Menton developed the model \[ \ce{E + S <=>[\ce{k_1^{+}}][k_1^{-}] C \xrightarrow{k_2} P + E} \] which fits experimental results much better.
By performing equilibrium analysis, we concluded that the rate of the reaction is \[ V = \frac{dp}{dt} = \frac{V_{max}\cdot [S]}{K_D + [S]} \]

However, this still assumed that $ \frac{d}{dt}[S] \approx 0 $...which is a problem.
If we assume that enzymes are primarily occupied with binding substrate, then instead we can set $ \frac{d}{dt}[C] \approx 0 $...that is, the concentration of the substrate/enzyme complex remains more or less consistent.
From this, we can derive the following.
\[ \frac{dc}{dt} = k_1^{+}[S][E] - (k_1^{-} + k_2)[C] \]
Since $\frac{dc}{dt} \approx 0$, we have
\[ k_1^{+}[S][E] = (k_1^{-} + k_2)[C] \]
\[ [C] = \frac{k_1^{+}}{k_1^{-} + k_2}[S][E] \]

We are interested in \[ V = \frac{dp}{dt} = k_2 [C] \] in terms of $[S]$.
\[ [C] = \frac{k_1^{+}}{k_1^{-} + k_2}[S](e_0 - [C]) = \frac{e_0[S]}{\frac{k_1^{+}}{k_1^{-} + k_2} + [S]} = \frac{e_0[S]}{K_M + [S]} \]

Therefore, \[ V = \frac{dp}{dt} = k_2\frac{e_0[S]}{K_M + [S]} = \frac{V_{max} \cdot [S]}{K_M + [S]} \]

\section*{Inhibition}
\begin{mitemize}
  \item competitive
  \item allosteric
\end{mitemize}
You can distinguish these two by changing the substrate concentration and observing the maximum reaction velocity (rate).

\subsection*{Competitive inhibition}
\[ \ce{E + S <=>[\ce{k_1^{+}}][k_1^{-}] C_1 \xrightarrow{k_2} P + E} \]
\[ \ce{E + I <=>[\ce{k_3^{+}}][k_3^{-}] C_2} \]

Let's assume $\frac{dc_1}{dt} \approx 0$ and $\frac{dc_2}{dt} \approx 0$.
\[ [C_1] = \frac{k_i e_0 [S]}{K_M [I] + k_i [S] + K_m k_i} \]
\[ [C_2] = \frac{K_m e_0 [I]}{K_M [I] + k_i [S] + K_m k_i} \]
\[ k_i = \frac{k_3^{-}}{k_3^{+}} \text{\hspace{25px}(dissociation constant of inhibitor)} \]
\[ k_m = \frac{k_2 + k_1^{-}}{k_1^{+}} \]
\[ V = \frac{dp}{dt} = k_2 [C_1]\frac{k_2 e_0 [S] k_i}{k_m [I] + k_i [S] + k_m k_i} = \frac{[S] V_{max}}{[S] + k_m\left(1 + \frac{[I]}{k_i}\right)} \]
\[ V = \frac{V_{max} \cdot [S]}{[S] + k_m\left(1 + \frac{[I]}{k_i}\right)} \]
If we saturate with substrate, we will still get maximum velocity eventually.

\subsection*{Allosteric inhibition}
\[ V = \frac{V_{max}}{1 + \frac{i}{k_i}}\cdot\frac{[S]}{k_m + [S]} \]
In allosteric inhibition, the maximum velocity is reduced.

\end{document}
