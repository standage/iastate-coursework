\documentclass[a4paper, 10pt]{article}
\usepackage[margin=1in]{geometry}
\usepackage[parfill]{parskip}          % skip a line instead of indenting new paragraphs
\usepackage{hyperref}
\hypersetup
{
  colorlinks=false,
  pdfborder={0,0,0},
}
 
\usepackage{fancyhdr}
\fancyhead[L]{\class \;- \assignment  \;- \duedate }
\fancyhead[R]{\author }
\renewcommand{\footrulewidth}{0.5pt} % Insert a line above the footer
\pagestyle{fancy}
\usepackage[hang,small,bf]{caption}
\usepackage{palatino}
\usepackage{amsmath}
\usepackage{amssymb}
\usepackage{enumerate}
\usepackage{minted}
 
% convenience commands
\renewcommand{\author}{Daniel Standage}
\newcommand{\class}{BCB 570}
\newcommand{\instructor}{Matzavinos}
\newcommand{\assignment}{Homework 4}
\newcommand{\duedate}{Apr 13, 2012}
 
\newcounter{prob_num}
\setcounter{prob_num}{1}
% usage: \problem
\newcommand{\problem}{\vspace{20pt}\arabic{prob_num}.\stepcounter{prob_num}\par}
% usage: \head{name}{class}{assignment}
\newcommand{\head}{\begin{center}\begin{tabular*}{\linewidth}{l@{\extracolsep{\fill}}r} & \class \;- \assignment \\ & \duedate \end{tabular*}\end{center} \hfill }
% usage: \eqn{equation}{label}
\newcommand{\eqn}[2]{\begin{equation}#1\label{#2}\end{equation}}
 
\begin{document}

Cast as a classification problem, the $k$-means clustering algorithm achieves a sensitivity of 0.613, a specificity of 0.997, and an F1 score of 0.758. The confusion matrix is as follows.
\begin{verbatim}

     -------------------------- Diagnosis --------------------------
     |
     |          tp = 130                     fp = 1
 Prediction
  (cluster)
     |          fn = 82                      tn = 356
     |
     ---------------------------------------------------------------

\end{verbatim}
Essentially, $k$-means does an excellent job not classifying benign patients as malignant, but it fails to classify many of the malignant patients as such.

\section*{Appendix}
The following Matlab script performs the k-means clustering.

\bigskip
\textbf{\texttt{run-kmeans.m}}
\begin{minted}[mathescape,
               linenos,
               numbersep=5pt,
               gobble=2,
               frame=lines,
               framesep=2mm]{matlab}
  x = dlmread('wdbc-values.data', ',');
  c = kmeans(x, 2);
  dlmwrite('wdbc-clusters.data', c);
  exit
\end{minted}
\bigskip

The following shell script can be used to download the data, process it, run the Matlab script, and compare the clustering results with actual diagnoses.

\bigskip
\textbf{\texttt{run-all.sh}}
\begin{minted}[mathescape,
               linenos,
               numbersep=5pt,
               gobble=2,
               frame=lines,
               framesep=2mm]{bash}
  #!/bin/bash
  DATASERVER=http://archive.ics.uci.edu
  DATAPATH=ml/machine-learning-databases/breast-cancer-wisconsin
  curl -o wdbc.data $DATASERVER/$DATAPATH/wdbc.data
  perl -ne '@f = split/,/; print(join(",", @f[2..31]))' < wdbc.data > wdbc-values.data
  perl -ne '@f = split/,/; printf("%s\n", $f[1])' < wdbc.data > wdbc-diagnoses.data
  /Applications/MATLAB_R2011b.app/bin/matlab -nodisplay < run-kmeans.m
  echo -e "\n\n\n======Results======"
  paste -d: wdbc-clusters.data wdbc-diagnoses.data | sort | uniq -c
\end{minted}
\bigskip

Running the shell script on my desktop gives the following terminal output.

\bigskip
\begin{verbatim}
dhrasmus:hw04 standage$ bash run-all.sh 
  % Total    % Received % Xferd  Average Speed   Time    Time     Time  Current
                                 Dload  Upload   Total   Spent    Left  Speed
100  121k  100  121k    0     0   131k      0 --:--:-- --:--:-- --:--:--  253k

                            < M A T L A B (R) >
                  Copyright 1984-2011 The MathWorks, Inc.
                    R2011b (7.13.0.564) 64-bit (maci64)
                              August 13, 2011

 
To get started, type one of these: helpwin, helpdesk, or demo.
For product information, visit www.mathworks.com.
 
>> >> >> >> 


======Results======
   1 1:B
 130 1:M
 356 2:B
  82 2:M
\end{verbatim}
\bigskip

\end{document}
