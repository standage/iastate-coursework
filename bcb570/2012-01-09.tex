\documentclass[10pt]{article}
\usepackage[margin=1in]{geometry}
\usepackage[shortlabels]{enumitem}
%\setcounter{secnumdepth}{0}
\usepackage{amssymb,amsmath,amsthm}
\usepackage{graphicx}
\usepackage{caption}
\usepackage{fancyhdr, lastpage}
\usepackage{mhchem}
\pagestyle{fancy}
\fancyhf{}
\lhead{Daniel Standage}
\chead{BCB 570, 3:10pm MWF}
\rhead{Lecture Notes: 9 Jan, 2012}
%\cfoot{Page \thepage{} of \protect\pageref*{LastPage}}
\usepackage{varioref}
\labelformat{equation}{(#1)}
\usepackage[colorlinks,linkcolor=blue]{hyperref}

\newenvironment{mitemize}
{
  \begin{itemize}
  \setlength{\itemsep}{1pt}
  \setlength{\parskip}{0pt}
  \setlength{\parsep}{0pt}}{\end{itemize}
}

\newenvironment{menumerate}
{
  \begin{enumerate}
  \setlength{\itemsep}{1pt}
  \setlength{\parskip}{0pt}
  \setlength{\parsep}{0pt}}{\end{enumerate}
}


\begin{document}

\section*{BCB 570}
\subsection*{Class info}
\begin{mitemize}
  \item Tasos (Anastasios) Matzavinos: tasos@iastate.edu
  \item Formal office hours after class, 4-5pm MWF (informally by appointment)
  \item Grading scheme: 40\% homeworks, 60\% class projects
\end{mitemize}

\subsection*{Class organization}

\subsubsection{Kinetic modeling of metabolic networks}
\begin{mitemize}
  \item deterministic and stochastic models
  \item structural analysis of networks
  \item extreme pathways
  \item flux cones
  \item XPP software
\end{mitemize}

\subsubsection{Analysis of high-throughput (genomic) data}
\begin{mitemize}
  \item clustering
  \item kernelized SVMs
\end{mitemize}

\subsubsection{Transcription networks}
\begin{mitemize}
  \item motifs
  \item random graphs
  \item scale-free networks
  \item regulatory networks
\end{mitemize}


\section*{Kinetic modeling: the dynamics of simple decay}
Consider the reaction \[ M \rightarrow \emptyset \] in which molecules of substance $M$ degrade into some substance we are not interested in tracking.
Let the function $M(t)$ represent the number of molecules of $M$ at time $t$.
Now, assume that every minute, 2 out of every 100 molecules of $M$ degrade.
Thus, the \textit{probability} $p_1$ of a molecule degrading within the time span of a minute is $\frac{2}{100} = \frac{1}{50} = 0.02 = p_1$.
The \textit{rate} $k_1$ of the reaction is the probability per unit time: in this case, $k_1 = \frac{1}{50}$ molecules per minute.

Let $p_n$ be the probability of degredation within $n$ minutes.
\[ p_2 = \frac{2}{100} + \frac{2}{100} = \frac{4}{100} = \frac{1}{25} \]
\[ k_2 = \frac{p_2}{2} = \frac{\frac{1}{25}}{2} = \frac{1}{50} = k_1 \]
Therefore, while the probability of degredation depends on time, the reaction rate does not.
We can write this reaction as a \textit{discrete time model}.
\[ M(t + \Delta t) = M(t) - p_{\Delta t}M(t) \]
\[ M(t + \Delta t) - M(t) = -p_{\Delta t}M(t) \]
\[ \frac{M(t + \Delta t) - M(t)}{\Delta t} = \frac{-p_{\Delta t}}{\Delta t}M(t) = -kM(t) \]
If we let $\Delta t \rightarrow 0$, we have the differential equation \[ \frac{d}{dt}M(t) = -kM(t) \] with solution \[ M(t) = M(0)e^{-kt} \]

\subsection*{Law of mass action}
This law is only applicable when molecular species are present in an abundance.
It is also only applicable to \textit{elementary reactions} (to be defined).

Consider a reaction \[ A + B \xrightarrow{k} C \] in which the rate $k$ refers to the accumulation of molecule $C$. We are interested in calculating the concentration of $C$ $[C]$. The Law of Mass Action identifies $\frac{d}{dt}[C]$ with the (mathematical) product $k[A][B]$.

Now consider the reversible extension of this reaction.
\[ \ce{A + B <=>[k^{+}][k^{-}] C} \]
If we want to measure $[A]$, we have the following differential equation.
\[ \frac{d}{dt}[A] = k^{-}[C] -\frac{d}{dt}[C] \]
\[ \frac{d}{dt}[A] = k^{-}[C] -k^{+}[A][B] \]

We will cover the Law of Mass Action in more detail during the next lecture.

\end{document}
