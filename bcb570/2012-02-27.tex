\documentclass[10pt]{article}
\usepackage[margin=1in]{geometry}
\usepackage[shortlabels]{enumitem}
%\setcounter{secnumdepth}{0}
\usepackage{amssymb,amsmath,amsthm}
\usepackage{graphicx}
\usepackage{caption}
\usepackage{fancyhdr, lastpage}
\usepackage{mhchem}
\pagestyle{fancy}
\fancyhf{}
\lhead{Daniel Standage}
\chead{BCB 570, 3:10pm MWF}
\rhead{Lecture Notes: 27 Feb, 2012}
%\cfoot{Page \thepage{} of \protect\pageref*{LastPage}}
\usepackage{varioref}
\labelformat{equation}{(#1)}
\usepackage[colorlinks,linkcolor=blue]{hyperref}

\newenvironment{mitemize}
{
  \begin{itemize}
  \setlength{\itemsep}{1pt}
  \setlength{\parskip}{0pt}
  \setlength{\parsep}{0pt}}{\end{itemize}
}

\newenvironment{menumerate}
{
  \begin{enumerate}
  \setlength{\itemsep}{1pt}
  \setlength{\parskip}{0pt}
  \setlength{\parsep}{0pt}}{\end{enumerate}
}


\begin{document}

\section*{Boolean networks}

A boolean network is a graph $G(V, F)$ consisting of a set of vertices $V$ (genes) and a list of boolean functions $F = {f_1, f_2, ..., f_n}$ where $f_i = {v_k1, v_k2, ..., v_kn}$.

An expression pattern is a function $\psi : V --> {0,1}$

One can define dynamics on a boolean network by the following scheme. \[ \psi_{t+1}(v_i) = f_i(\psi_t(v_i1), \psi_t(v_i2), ..., \psi_t(v_ik)) \]

Consider a dataset of the form \[ D = {(I_1, O_1), (I_2, O_2), ..., (I_m, O_m)} \] where $I_k$ and $O_k$ are expression profiles.

\subsection*{Definitions}
\begin{menumerate}
  \item We say that a node $v_i$ in a boolean network is consistent with $(I_j, O_j)$ if \[ O_j(v_i) = f_i(I_j(v_i1), ..., I_j(V_ik)) \]
  \item We say $G(V, F)$ is consistent with $(I_j, O_j)$ if all nodes are consistent
\end{menumerate}

Given $n$ (the number of genes) and data $D$, decide whether or not there exists a boolean network $G(F,V)$ consistent with $D$, and output one if it exists

\end{document}
