\documentclass[10pt]{article}
\usepackage[margin=1in]{geometry}
\usepackage[shortlabels]{enumitem}
%\setcounter{secnumdepth}{0}
\usepackage{amssymb,amsmath,amsthm}
\usepackage{graphicx}
\usepackage{caption}
\usepackage{fancyhdr, lastpage}
\pagestyle{fancy}
\fancyhf{}
\lhead{Daniel Standage}
\chead{GDCB 511, 9:00am MWF}
\rhead{Lecture Notes: 11 Jan, 2012}
%\cfoot{Page \thepage{} of \protect\pageref*{LastPage}}
\usepackage{varioref}
\labelformat{equation}{(#1)}
\usepackage[colorlinks,linkcolor=blue]{hyperref}

\newenvironment{mitemize}
{
  \begin{itemize}
  \setlength{\itemsep}{1pt}
  \setlength{\parskip}{0pt}
  \setlength{\parsep}{0pt}}{\end{itemize}
}

\newenvironment{menumerate}
{
  \begin{enumerate}
  \setlength{\itemsep}{1pt}
  \setlength{\parskip}{0pt}
  \setlength{\parsep}{0pt}}{\end{enumerate}
}


\begin{document}

\section*{2D Protein Gels}
\begin{mitemize}
  \item separate proteins in one direction by charge, and then the other direction by size
  \item common approach is to make two protein preparations (for example, presence and absence of some substance), each with a different colored stain (red and blue)
  \item after separation, most protein bands will be purple, but blue and red bands indicate proteins that are effected by the substance
\end{mitemize}

\section*{Protein-DNA interactions}

\subsection*{Gel mobility shift assay}
DNA-bound proteins will move more slowly through the electrophoresis medium, causing a band shift.

\subsection*{DNAseI Footprinting}
\begin{mitemize}
  \item brief digestion
  \item run on a gel with nucleotide resolution
  \item bands corresponding to protein-bound nucleotides will not show up
\end{mitemize}

\subsection*{ChIP}
Detection of \textit{in vivo} interactions


\section*{Protein-Protein Interactions}

\subsection*{Yeast 2-hybrid system}
\begin{mitemize}
  \item transcription factors can be broken down into two components; when those components bind, transcription of reporter gene occurs
  \item attach two proteins of interest to the two components of the transcription factor---one protein to one component, one protein to the other
  \item if the proteins of interest do not interact and bind, no transcription will occur and no reporter will be observed; however, if they do interact, then the transcription factor will be complete and the reporter gene will be transcribed and observed
\end{mitemize}

\subsection*{GST pull-down}
\subsection*{Co-IP}

\end{document}
