\documentclass[10pt]{article}
\usepackage[margin=1in]{geometry}
\usepackage[shortlabels]{enumitem}
%\setcounter{secnumdepth}{0}
\usepackage{amssymb,amsmath,amsthm}
\usepackage{graphicx}
\usepackage{caption}
\usepackage{fancyhdr, lastpage}
\pagestyle{fancy}
\fancyhf{}
\lhead{Daniel Standage}
\chead{GDCB 511, 9:00am MWF}
\rhead{Lecture Notes: 3, 6 Feb, 2012}
%\cfoot{Page \thepage{} of \protect\pageref*{LastPage}}
\usepackage{varioref}
\labelformat{equation}{(#1)}
\usepackage[colorlinks,linkcolor=blue]{hyperref}

\newenvironment{mitemize}
{
  \begin{itemize}
  \setlength{\itemsep}{1pt}
  \setlength{\parskip}{0pt}
  \setlength{\parsep}{0pt}}{\end{itemize}
}

\newenvironment{menumerate}
{
  \begin{enumerate}
  \setlength{\itemsep}{1pt}
  \setlength{\parskip}{0pt}
  \setlength{\parsep}{0pt}}{\end{enumerate}
}

\newcommand{\textsup}[1]{\ensuremath{^{\textrm{#1}}}}
\newcommand{\textsub}[1]{\ensuremath{_{\textrm{#1}}}}


\begin{document}

\textbf{linker scanning}: make a new mutant for each 10 bp stretch (switch out those 10 nucleotides); observe phenotypic effects

\textbf{reading assignment}: Kostrewa et al, Nature 2009
\begin{mitemize}
  \item B reader: identify TSS
  \item B linker: open (melt) promoter
  \item H: helicase and kinase activity
  \item F: bind RNAPolII, interact with non-template strand
  \item H: phosphorylates RNAPolII CTD (C-terminal domain)
  \item E: stimulates H kinase activity
  \item S: stimulates proofreading and correction of transcripts
\end{mitemize}

We will not cover TF classes I and III in detail.

\section*{Function of TFIID}

\subsection*{TBP (TAT box-binding protein)}
\begin{mitemize}
  \item binds TAT box at minor groove
  \item bends DNA to start transcription initiation
\end{mitemize}

\subsection*{TAF\textsub{II}s (TBP associated factors)}
\begin{mitemize}
  \item recruit TBP to TATA-less promoter to start transcription initiation (TAF\textsub{II}250 and TAF\textsub{II}150 bind to initiator and DPE, TAF\textsub{II}250 and TAF\textsub{II}110 interact with Sp1 that binds to GC box)
  \item different TAF\textsub{II}s are required to respond to various activators
  \item TAF\textsub{II}250 has histone acetyltransferase (HAT) and kinase activities that modify chromatin and other transcription factors
\end{mitemize}

\section*{Promoter proximal pausing}
\begin{mitemize}
  \item pause sites 20-50bp downstream of TSS
  \item two proteins help stabilize RNA PolII in paused state: DSIF and NELF
  \item P-TEFb delivers signal to leave paused state
\end{mitemize}

\end{document}
