\documentclass[10pt]{article}
\usepackage[margin=1in]{geometry}
\usepackage[shortlabels]{enumitem}
%\setcounter{secnumdepth}{0}
\usepackage{amssymb,amsmath,amsthm}
\usepackage{graphicx}
\usepackage{caption}
\usepackage{fancyhdr, lastpage}
\pagestyle{fancy}
\fancyhf{}
\lhead{Daniel Standage}
\chead{GDCB 511, 9:00am MWF}
\rhead{Lecture Notes: 23 Jan, 2012}
%\cfoot{Page \thepage{} of \protect\pageref*{LastPage}}
\usepackage{varioref}
\labelformat{equation}{(#1)}
\usepackage[colorlinks,linkcolor=blue]{hyperref}

\newenvironment{mitemize}
{
  \begin{itemize}
  \setlength{\itemsep}{1pt}
  \setlength{\parskip}{0pt}
  \setlength{\parsep}{0pt}}{\end{itemize}
}

\newenvironment{menumerate}
{
  \begin{enumerate}
  \setlength{\itemsep}{1pt}
  \setlength{\parskip}{0pt}
  \setlength{\parsep}{0pt}}{\end{enumerate}
}


\begin{document}

\section*{Temporal control of transcription in phage infection}
\begin{mitemize}
  \item early genes recognized by intrinsic $\sigma$ factor (-10 and -35 elements)
  \item early genes are transcription factors for middle genes, which in turn are TFs for late genes
\end{mitemize}

\subsection*{T7 phage}
\begin{itemize}
  \item host polymerase transcribes class I genes
  \item one of the class I genes is the phage polymerase, which synthesized class II and III genes
\end{itemize}

\subsection*{$\lambda$ phage infection}
\begin{itemize}
  \item 2 competing phases: lytic and lysogenic
  \item cI gene encodes a $\lambda$ repressor, induces lysogenic phase
  \item cro gene induces lytic phase
\end{itemize}

\end{document}
