\documentclass[10pt]{article}
\usepackage[margin=1in]{geometry}
\usepackage[shortlabels]{enumitem}
%\setcounter{secnumdepth}{0}
\usepackage{amssymb,amsmath,amsthm}
\usepackage{graphicx}
\usepackage{caption}
\usepackage{fancyhdr, lastpage}
\pagestyle{fancy}
\fancyhf{}
\lhead{Daniel Standage}
\chead{GDCB 511, 9:00am MWF}
\rhead{Lecture Notes: 9 Jan, 2012}
%\cfoot{Page \thepage{} of \protect\pageref*{LastPage}}
\usepackage{varioref}
\labelformat{equation}{(#1)}
\usepackage[colorlinks,linkcolor=blue]{hyperref}

\newenvironment{mitemize}
{
  \begin{itemize}
  \setlength{\itemsep}{1pt}
  \setlength{\parskip}{0pt}
  \setlength{\parsep}{0pt}}{\end{itemize}
}

\newenvironment{menumerate}
{
  \begin{enumerate}
  \setlength{\itemsep}{1pt}
  \setlength{\parskip}{0pt}
  \setlength{\parsep}{0pt}}{\end{enumerate}
}


\begin{document}

\section*{GDCB 511}
\subsection*{Course notes}
\begin{mitemize}
  \item TA: Rebecca Weeks (rlmauton@iastate.edu)
  \item Purpose of textbook: learn how to draw conclusions from experimental results
\end{mitemize}

\subsection*{Central dogma}
Our course will focus on the central dogma of molecular biology; specifically, we will look at:
\begin{mitemize}
  \item the imporant processes in molecular detail
  \item the related regulatory mechanisms
\end{mitemize}


\section*{Biomolecular structures}
\subsection*{DNA structure}
\begin{mitemize}
  \item phosphodiester bonds between 5' phosphate and 3' carbon
  \item double helical structure
  \begin{mitemize}
    \item most binding on major groove
    \item .34 nm (3.4 \AA) between nucleotides
    \item 3.4 nm (34 \AA) period in the helix (10 bp = 1 turn of the helix)
  \end{mitemize}
\end{mitemize}

\subsection*{Protein structure}
\begin{mitemize}
  \item I need to memorize the peptide bond
  \item reactivity at the peptide bond (especially the double-bonded oxygen and the hydrogen) is responsible for protein secondary structure
  \item secondary structure elements: $\alpha$ helix, $\beta$ sheet, loops (especially near prolines)
\end{mitemize}


\section*{Molecular cloning}
\begin{mitemize}
  \item molecular recombination using restriction enzymes, DNA ligase
  \item alkaline phosphatase method can be used for screening out vectors that have re-ligated to themselves (vs vectors successfully transformed with recombinant DNA)
\end{mitemize}

\end{document}
